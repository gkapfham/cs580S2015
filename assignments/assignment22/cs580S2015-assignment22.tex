%!TEX root=cs580S2015-assignment22.tex
% mainfile: cs580S2015-assignment22.tex

%!TEX root=cs580S2015-syllabus.tex
% mainfile: cs580S2015-syllabus.tex 

% CS 580 style
% Typical usage (all UPPERCASE items are optional):
%       \input 580pre
%       \begin{document}
%       \MYTITLE{Title of document, e.g., Lab 1\\Due ...}
%       \MYHEADERS{short title}{other running head, e.g., due date}
%       \PURPOSE{Description of purpose}
%       \SUMMARY{Very short overview of assignment}
%       \DETAILS{Detailed description}
%         \SUBHEAD{if needed} ...
%         \SUBHEAD{if needed} ...
%          ...
%       \HANDIN{What to hand in and how}
%       \begin{checklist}
%       \item ...
%       \end{checklist}
% There is no need to include a "\documentstyle."
% However, there should be an "\end{document}."
%
%===========================================================

\documentclass[11pt,twoside,titlepage]{article}

\usepackage{threeparttop}
\usepackage{graphicx}
\usepackage{latexsym}
\usepackage{color}
\usepackage{listings}
\usepackage{fancyvrb}
%\usepackage{pgf,pgfarrows,pgfnodes,pgfautomata,pgfheaps,pgfshade}
\usepackage{tikz}
\usepackage[normalem]{ulem}
\tikzset{
    %Define standard arrow tip
%    >=stealth',
    %Define style for boxes
    oval/.style={
           rectangle,
           rounded corners,
           draw=black, very thick,
           text width=6.5em,
           minimum height=2em,
           text centered},
    % Define arrow style
    arr/.style={
           ->,
           thick,
           shorten <=2pt,
           shorten >=2pt,}
}
\usepackage[noend]{algorithmic}
\usepackage[noend]{algorithm}
\newcommand{\bfor}{{\bf for\ }}
\newcommand{\bthen}{{\bf then\ }}
\newcommand{\bwhile}{{\bf while\ }}
\newcommand{\btrue}{{\bf true\ }}
\newcommand{\bfalse}{{\bf false\ }}
\newcommand{\bto}{{\bf to\ }}
\newcommand{\bdo}{{\bf do\ }}
\newcommand{\bif}{{\bf if\ }}
\newcommand{\belse}{{\bf else\ }}
\newcommand{\band}{{\bf and\ }}
\newcommand{\breturn}{{\bf return\ }}
\newcommand{\mod}{{\rm mod}}
\renewcommand{\algorithmiccomment}[1]{$\rhd$ #1}
\newenvironment{checklist}{\par\noindent\hspace{-.25in}{\bf Checklist:}\renewcommand{\labelitemi}{$\Box$}%
\begin{itemize}}{\end{itemize}}
\pagestyle{threepartheadings}
\usepackage{url}
\usepackage{wrapfig}
% removing the standard hyperref to avoid the horrible boxes
%\usepackage{hyperref}
\usepackage[hidelinks]{hyperref}
% added in the dtklogos for the bibtex formatting
\usepackage{dtklogos}
%=========================
% One-inch margins everywhere
%=========================
\setlength{\topmargin}{0in}
\setlength{\textheight}{8.5in}
\setlength{\oddsidemargin}{0in}
\setlength{\evensidemargin}{0in}
\setlength{\textwidth}{6.5in}
%===============================
%===============================
% Macro for document title:
%===============================
\newcommand{\MYTITLE}[1]%
   {\begin{center}
     \begin{center}
     \bf
     CMPSC 580\\Topics and Research Methods in Computer Science\\
     Spring 2014
     \medskip
     \end{center}
     \bf
     #1
     \end{center}
}
%================================
% Macro for headings:
%================================
\newcommand{\MYHEADERS}[2]%
   {\lhead{#1}
    \rhead{#2}
    %\immediate\write16{}
    %\immediate\write16{DATE OF HANDOUT?}
    %\read16 to \dateofhandout
    \def \dateofhandout {January 14, 2014}
    \lfoot{\sc Handed out on \dateofhandout}
    %\immediate\write16{}
    %\immediate\write16{HANDOUT NUMBER?}
    %\read16 to\handoutnum
    \def \handoutnum {1}
    \rfoot{Handout \handoutnum}
   }

%================================
% Macro for bold italic:
%================================
\newcommand{\bit}[1]{{\textit{\textbf{#1}}}}

%=========================
% Non-zero paragraph skips.
%=========================
\setlength{\parskip}{1ex}

%=========================
% Create various environments:
%=========================
\newcommand{\PURPOSE}{\par\noindent\hspace{-.25in}{\bf Purpose:\ }}
\newcommand{\SUMMARY}{\par\noindent\hspace{-.25in}{\bf Summary:\ }}
\newcommand{\DETAILS}{\par\noindent\hspace{-.25in}{\bf Details:\ }}
\newcommand{\HANDIN}{\par\noindent\hspace{-.25in}{\bf Hand in:\ }}
\newcommand{\SUBHEAD}[1]{\bigskip\par\noindent\hspace{-.1in}{\sc #1}\\}
%\newenvironment{CHECKLIST}{\begin{itemize}}{\end{itemize}}

\usepackage[compact]{titlesec}

\begin{document}

\MYTITLE{Assignment 22\\Ideas for a Senior Thesis Proposal\\
Part 1 due 9 April 2015\\
Part 2 due 10 April 2015}
\MYHEADERS{Assignment 22}{Due Thu., Fri., 9--10 Apr.}

\subsection*{Introduction}

Due to schedule constraints brought on, at least in part, by the recent Gator Day during which we did not have class, I
have slightly adjusted the schedule for module four. For this module, you will not be asked to complete the paper review
or conceptual skills assignments, instead focusing on preparing for a senior thesis proposal related to this module's
theme.

\subsection*{Part 1: Five Ideas}

Create a short \LaTeX\ document listing (at least) five possible senior thesis ideas that are related in some way to the
topic of autonomous multi-robot systems. For each idea, come up with a proposed thesis title and a one- or two-sentence
hypothesis/thesis statement, along with a description of your evaluation metrics and a short overview of the system that
you plan to implement. Hand in a hard copy of this list at the beginning of Thursday morning's class (you might want to
keep a copy for yourself to refer to during Thursday's discussion). Make sure you keep it in your Git repository.

These do {\em not} have to be ``tightly coupled'' to the subject of multi-robot systems that complete tasks such as
perimeter surveillance and landmine detection. They do {\em not} have to be of the form ``A new multi-robot system to
handle \ldots'' or ``A novel empirical study of a robotic system \ldots.'' Yet, there should be a clear connection
between your proposed idea and the topic of this module.

It would be ideal if you come up with ideas that are personally of interest to you, but that is not an absolute
requirement. Just try to propose feasible ideas; even if you are not interested in carrying out such a project, perhaps
someone else in the class might find your idea worth pursuing.

During a session that a student will chair, we will discuss these ideas in class. Be prepared to provide further
information about your ideas during this in-class discussion.

\subsection*{Part 2: Lightning Talk and Slides}

For Friday afternoon, prepare a ``lightning talk'', lasting about 3--5 minutes, together with several accompanying
Beamer slides, in which you expand upon an idea for a senior thesis related in some way to multi-robot systems. In
advance of Friday's session, please practice your talk to ensure that you can complete it in no more than five minutes;
you should review your proposed topic so that you are ready to answer several questions about how you will pursue it for
your senior thesis.

In addition to placing the \LaTeX\ source code of your slides in your Git repository, you must submit signed and printed
copies of your slides at the beginning of Friday's class. Students who would like to complete sophisticated Beamer
slides should search online for tutorials about creating their own ``inner'', ``outer'', and ``color'' themes. If you
would like to apply existing styles to your slides, then please visit \url{http://deic.uab.es/~iblanes/beamer_gallery/}
for many examples.

\end{document}
