%!TEX root=cs580S2014-assignment3.tex
% mainfile: cs580S2014-assignment3.tex 
\input{580pre.tex}
\usepackage[compact]{titlesec}

\begin{document}

\MYTITLE{Assignment 3: In-Class Exercise\\21 January 2014\\Writing Good Proposals}
\MYHEADERS{Writing Good Proposals}{}

\subsection*{Background}
The most important document you will produce in the Junior Seminar is a
proposal for your senior project. Today we take a first look at proposals.

A good proposal should include:
\begin{itemize}
\item
compelling motivation for the proposed project
through examples, published data, demonstrated need, or other evidence
\item
proof that the proposer understands the problem and is 
thoroughly acquainted
with the scholarly literature surrounding it
\item
an approach to carrying out the project that is well-considered, 
professional, and described in sufficient detail that the reader is able to
assess its feasibility
\item
a realistic timetable for
completing the major components of the project
\end{itemize}
Other sections may be included on
threats to validity (what objections might there be
to your method of approach and how would you address them?), 
specialized computational needs (for example, do you need to reserve
time on some of the Department lab machines?), budget (does your project
require the purchase of special software or hardware?), a back-up plan (in
case your project can not be completed as proposed), etc.

\subsection*{Evaluate Some Proposals}
Work in groups of three. Each group will receive a set of six proposals. 
These are actual proposals submitted by Allegheny College Computer 
Science majors; names and dates have been blacked out.

The proposals represent all stages of the proposal-writing process,
from preliminary proposals produced in CMPSC 580, through initial drafts
of proposals submitted in CMPSC 600, to final proposals accepted by
the faculty. 

Evaluate each of the proposals on each of the above criteria. Note any
aspects of the proposals that need improvement. Note any aspects that
are particularly well-done. At the end of class, the groups will be asked
to report back with answers to the following questions:
\begin{enumerate}
\item
For each proposal 1 through 6, does the author provide sufficient background
and motivation for the project?
\item
For each proposal, does the author demonstrate proficiency in the area
of the proposed project? How is this achieved?
\item
For each proposal, are there figures, tables, graphs, or other
visual aids to help clarify the concepts being discussed? Do the
figures appear to be from outside sources? If so, are the
sources properly acknowledged and are the figures used with permission?
\item
For each proposal, does the author demonstrate familiarity with the
scholarly literature surrounding the proposed project? Do the
references seem complete? Are the references primarily to Web sites? To
books? To journals?
\item
For each proposal, is it clear how the proposer intends to carry out
the research and what the final deliverables will look like?
\item
For each proposal, does it seem to be feasible as a senior project at
Allegheny College?
\item
For each proposal, highlight any features that your group found to be
particularly well-done. Highlight any features that need improvement.
Feel free to make general comments concerning the proposals.

\end{enumerate}



\end{document}
