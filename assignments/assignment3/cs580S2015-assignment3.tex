%!TEX root=cs580S2015-assignment3.tex
% mainfile: cs580S2015-assignment3.tex 
%!TEX root=cs580S2015-syllabus.tex
% mainfile: cs580S2015-syllabus.tex 

% CS 580 style
% Typical usage (all UPPERCASE items are optional):
%       \input 580pre
%       \begin{document}
%       \MYTITLE{Title of document, e.g., Lab 1\\Due ...}
%       \MYHEADERS{short title}{other running head, e.g., due date}
%       \PURPOSE{Description of purpose}
%       \SUMMARY{Very short overview of assignment}
%       \DETAILS{Detailed description}
%         \SUBHEAD{if needed} ...
%         \SUBHEAD{if needed} ...
%          ...
%       \HANDIN{What to hand in and how}
%       \begin{checklist}
%       \item ...
%       \end{checklist}
% There is no need to include a "\documentstyle."
% However, there should be an "\end{document}."
%
%===========================================================

\documentclass[11pt,twoside,titlepage]{article}

\usepackage{threeparttop}
\usepackage{graphicx}
\usepackage{latexsym}
\usepackage{color}
\usepackage{listings}
\usepackage{fancyvrb}
%\usepackage{pgf,pgfarrows,pgfnodes,pgfautomata,pgfheaps,pgfshade}
\usepackage{tikz}
\usepackage[normalem]{ulem}
\tikzset{
    %Define standard arrow tip
%    >=stealth',
    %Define style for boxes
    oval/.style={
           rectangle,
           rounded corners,
           draw=black, very thick,
           text width=6.5em,
           minimum height=2em,
           text centered},
    % Define arrow style
    arr/.style={
           ->,
           thick,
           shorten <=2pt,
           shorten >=2pt,}
}
\usepackage[noend]{algorithmic}
\usepackage[noend]{algorithm}
\newcommand{\bfor}{{\bf for\ }}
\newcommand{\bthen}{{\bf then\ }}
\newcommand{\bwhile}{{\bf while\ }}
\newcommand{\btrue}{{\bf true\ }}
\newcommand{\bfalse}{{\bf false\ }}
\newcommand{\bto}{{\bf to\ }}
\newcommand{\bdo}{{\bf do\ }}
\newcommand{\bif}{{\bf if\ }}
\newcommand{\belse}{{\bf else\ }}
\newcommand{\band}{{\bf and\ }}
\newcommand{\breturn}{{\bf return\ }}
\newcommand{\mod}{{\rm mod}}
\renewcommand{\algorithmiccomment}[1]{$\rhd$ #1}
\newenvironment{checklist}{\par\noindent\hspace{-.25in}{\bf Checklist:}\renewcommand{\labelitemi}{$\Box$}%
\begin{itemize}}{\end{itemize}}
\pagestyle{threepartheadings}
\usepackage{url}
\usepackage{wrapfig}
% removing the standard hyperref to avoid the horrible boxes
%\usepackage{hyperref}
\usepackage[hidelinks]{hyperref}
% added in the dtklogos for the bibtex formatting
\usepackage{dtklogos}
%=========================
% One-inch margins everywhere
%=========================
\setlength{\topmargin}{0in}
\setlength{\textheight}{8.5in}
\setlength{\oddsidemargin}{0in}
\setlength{\evensidemargin}{0in}
\setlength{\textwidth}{6.5in}
%===============================
%===============================
% Macro for document title:
%===============================
\newcommand{\MYTITLE}[1]%
   {\begin{center}
     \begin{center}
     \bf
     CMPSC 580\\Topics and Research Methods in Computer Science\\
     Spring 2014
     \medskip
     \end{center}
     \bf
     #1
     \end{center}
}
%================================
% Macro for headings:
%================================
\newcommand{\MYHEADERS}[2]%
   {\lhead{#1}
    \rhead{#2}
    %\immediate\write16{}
    %\immediate\write16{DATE OF HANDOUT?}
    %\read16 to \dateofhandout
    \def \dateofhandout {January 14, 2014}
    \lfoot{\sc Handed out on \dateofhandout}
    %\immediate\write16{}
    %\immediate\write16{HANDOUT NUMBER?}
    %\read16 to\handoutnum
    \def \handoutnum {1}
    \rfoot{Handout \handoutnum}
   }

%================================
% Macro for bold italic:
%================================
\newcommand{\bit}[1]{{\textit{\textbf{#1}}}}

%=========================
% Non-zero paragraph skips.
%=========================
\setlength{\parskip}{1ex}

%=========================
% Create various environments:
%=========================
\newcommand{\PURPOSE}{\par\noindent\hspace{-.25in}{\bf Purpose:\ }}
\newcommand{\SUMMARY}{\par\noindent\hspace{-.25in}{\bf Summary:\ }}
\newcommand{\DETAILS}{\par\noindent\hspace{-.25in}{\bf Details:\ }}
\newcommand{\HANDIN}{\par\noindent\hspace{-.25in}{\bf Hand in:\ }}
\newcommand{\SUBHEAD}[1]{\bigskip\par\noindent\hspace{-.1in}{\sc #1}\\}
%\newenvironment{CHECKLIST}{\begin{itemize}}{\end{itemize}}

\usepackage[compact]{titlesec}

\begin{document}

\MYTITLE{Assignment 3: In-Class Exercise\\20 January 2015\\Writing Good Proposals}
\MYHEADERS{Writing Good Proposals}{}

\subsection*{Background}
The most important document you will produce in the Junior Seminar is a
proposal for your senior project. Today we take a first look at
proposals previously written by Allegheny College students.

A good proposal should include:
\begin{itemize}
\item
compelling motivation for the proposed project
through examples, published data, demonstrated need, or other evidence
\item
proof that the proposer understands the problem and is 
thoroughly acquainted
with the scholarly literature surrounding it
\item
an approach to carrying out the project that is well-considered, 
professional, and described in sufficient detail that the reader is able to
assess its feasibility
\item
a realistic timetable for
completing the major components of the project
\end{itemize}
Other sections may be included on
threats to validity (what objections might there be
to your method of approach and how would you address them?), 
specialized computational needs (for example, do you need to reserve
time on some of the Department lab machines?), budget (does your project
require the purchase of special software or hardware?), a back-up plan (in
case your project can not be completed as proposed), or other relevant
components of a proposal.

\subsection*{Evaluate Some Proposals}
Work in groups of three. Each group will repeatedly review a set of
three proposals that are set up at separate stations around the room.
These are actual proposals submitted by Allegheny College Computer 
Science majors; names and dates have been blacked out.

The proposals represent all stages of the proposal-writing process,
from preliminary proposals produced in CMPSC 580, through initial drafts
of proposals submitted in CMPSC 600, to final proposals accepted by
the faculty after a formal defense. 

Evaluate each of the proposals on each of the above criteria. Note any
aspects of the proposals that need improvement. Note any aspects that
are particularly well-done. At the end of class, the groups will be asked
to report back with answers to the following questions:
\begin{enumerate}
\item
For each proposal 1 through 6, does the author provide sufficient background
and motivation for the project?
\item
For each proposal, does the author demonstrate proficiency in the area
of the proposed project? How is this achieved?
\item
For each proposal, are there figures, tables, graphs, or other
visual aids to help clarify the concepts being discussed? Do the
figures appear to be from outside sources? If so, are the
sources properly acknowledged and are the figures used with permission?
\item
For each proposal, does the author demonstrate familiarity with the
scholarly literature surrounding the proposed project? Do the
references seem complete? Are the references primarily to Web sites? To
books? To journals? To articles in conference proceedings?
\item
For each proposal, is it clear how the proposer intends to carry out
the research and what the final deliverables (e.g., programs and data sets) will look like?
\item
For each proposal, does it seem to be feasible as a senior project at
Allegheny College?
\item
For each proposal, highlight any features that your group found to be
particularly well-done. Highlight any features that need improvement.
While remembering the importance of writing and speaking in a fashion
that is constructively critical, feel free to make additional comments concerning the proposals.

\end{enumerate}



\end{document}
