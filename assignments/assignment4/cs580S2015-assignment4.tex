%!TEX root=cs580S2014-assignment4.tex
% mainfile: cs580S2014-assignment4.tex
\input{580pre.tex}
\usepackage[compact]{titlesec}

\begin{document}

\MYTITLE{Assignment 4: In-Class Exercise\\22 January 2015\\Writing Theses}
\MYHEADERS{Writing Good Theses}{}

\subsection*{Thesis}

Today we examine what a thesis looks like and how to write a good one. Unlike a proposal, a thesis is a complete
documentation of your \textcolor{black}{finished} project. Everything you proposed to do should be completed at this
point in your research project. In a way, thesis is an extension of your proposal, as it should also describe and
motivate your research problem, describe scholarly literature surrounding it and propose your model/method for solving
the proposed problem.

However, a thesis should provide a more comprehensive description of the literature and a more detailed explanation of
the method you used---complete with, for instance, examples, diagrams, and an explanation of why it is appropriate. In
addition, it should also include sections describing your experiments (if any) and your results, and an involved
analysis of your experimental results. Finally, the thesis should conclude with a concise summary of the project, its
results and contributions, and ultimately propose possible future extensions for the project.

In summary, your thesis will be considered ``good'' if it includes:
\begin{itemize}
\item
\textcolor{black}{Compelling motivation for your completed project.} What are the research questions? Why is this project important, what does it contribute?
\item
\textcolor{black}{A comprehensive related work summary, integrating the thesis with the appropriate background.}
You should demonstrate your command and comprehension of the appropriate literature, including the most important publications in the field, recent research results on the subject, as well as modern summaries of the subject found in textbooks. The literature should be well-chosen and cited and referenced correctly (following the instructions).
\item
\textcolor{black}{A full explanation of the methods used in your project.}
You should demonstrate that the methods are used competently and that they are appropriate for answering your 
research questions. If any new method or an extension to a method is developed in the thesis, it should be clearly
reported; a methodological development could be a main focus of your thesis. The methods are described completely, so
that another researcher could repeat \mbox{the study}.
iitem
\textcolor{black}{A detailed explanation of the experimental set up, the experiments that were performed, and the results that were obtained.}
The results of your experiments should provide answers to the presented research questions, and the results should be
reported so that another researcher can evaluate their reliability and correctness.
\item
\textcolor{black}{The discussion and conclusions should reflect the results that were obtained in the study.} In the discussion, you should answer the proposed research questions and consider how the results of your project change or complement what was earlier known about the subject from the literature. In the conclusion, some possibilities for the future extension of this project should be discussed and possible new questions can be suggested.
\item
No precise limits on length are given for the thesis as a whole or for its individual part; lengths should be appropriate for the research subject and questions under study.
\end{itemize}

\subsection*{Evaluate Some Past Students' Theses}
Once again, please work in groups of three. We have five stations with one thesis in each station.
These are final, bound copies of theses completed by Allegheny College Computer
Science majors. Your group will start at one station and cover every other station.

Evaluate each of the theses on each of the above criteria. You do not need to read the whole thesis, just browse through it. Note any
aspects of the theses that need improvement. Note any aspects that
are particularly well-done. Your group should prepare a written response to each of these questions for each of the
theses that you review. In the second half of the class, the groups will be asked
to report back with answers to the following questions:
\begin{enumerate}
\item
For each thesis, does the author provide sufficient background
and motivation for the project?
\item
For each thesis, does the author demonstrate proficiency in the area
of the completed project? What methods does the author select as a means of showing proficiency?
\item
For each thesis, does the author demonstrate familiarity with the
scholarly literature surrounding the completed project? Do the
references seem complete? Are the references primarily to Web sites? To
books? To journals? To papers in conference proceedings?
\item
For each thesis, are there figures, tables, graphs, diagrams, or other
visual aids to help clarify the concepts being discussed? Do the
figures appear to be from outside sources? If so, are the
sources properly acknowledged and are the figures used with permission?
\item
For each thesis, is it clear how the author carried out
the research? Are the methods and experiments, if any, explained clearly and completely?
\item
For each thesis, does the author provide insightful discussions and draw valid conclusions from the results of the
experiments and explain the main contributions of the completed project?
\item
For each thesis, highlight any features that your group found to be
particularly well-done or that need improvement.
Feel free to make general comments concerning theses.
\end{enumerate}



\end{document}
