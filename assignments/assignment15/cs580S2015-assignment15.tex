%!TEX root=cs580S2014-assignment24.tex
% mainfile: cs580S2014-assignment24.tex 
%!TEX root=cs580S2015-syllabus.tex
% mainfile: cs580S2015-syllabus.tex 

% CS 580 style
% Typical usage (all UPPERCASE items are optional):
%       \input 580pre
%       \begin{document}
%       \MYTITLE{Title of document, e.g., Lab 1\\Due ...}
%       \MYHEADERS{short title}{other running head, e.g., due date}
%       \PURPOSE{Description of purpose}
%       \SUMMARY{Very short overview of assignment}
%       \DETAILS{Detailed description}
%         \SUBHEAD{if needed} ...
%         \SUBHEAD{if needed} ...
%          ...
%       \HANDIN{What to hand in and how}
%       \begin{checklist}
%       \item ...
%       \end{checklist}
% There is no need to include a "\documentstyle."
% However, there should be an "\end{document}."
%
%===========================================================

\documentclass[11pt,twoside,titlepage]{article}

\usepackage{threeparttop}
\usepackage{graphicx}
\usepackage{latexsym}
\usepackage{color}
\usepackage{listings}
\usepackage{fancyvrb}
%\usepackage{pgf,pgfarrows,pgfnodes,pgfautomata,pgfheaps,pgfshade}
\usepackage{tikz}
\usepackage[normalem]{ulem}
\tikzset{
    %Define standard arrow tip
%    >=stealth',
    %Define style for boxes
    oval/.style={
           rectangle,
           rounded corners,
           draw=black, very thick,
           text width=6.5em,
           minimum height=2em,
           text centered},
    % Define arrow style
    arr/.style={
           ->,
           thick,
           shorten <=2pt,
           shorten >=2pt,}
}
\usepackage[noend]{algorithmic}
\usepackage[noend]{algorithm}
\newcommand{\bfor}{{\bf for\ }}
\newcommand{\bthen}{{\bf then\ }}
\newcommand{\bwhile}{{\bf while\ }}
\newcommand{\btrue}{{\bf true\ }}
\newcommand{\bfalse}{{\bf false\ }}
\newcommand{\bto}{{\bf to\ }}
\newcommand{\bdo}{{\bf do\ }}
\newcommand{\bif}{{\bf if\ }}
\newcommand{\belse}{{\bf else\ }}
\newcommand{\band}{{\bf and\ }}
\newcommand{\breturn}{{\bf return\ }}
\newcommand{\mod}{{\rm mod}}
\renewcommand{\algorithmiccomment}[1]{$\rhd$ #1}
\newenvironment{checklist}{\par\noindent\hspace{-.25in}{\bf Checklist:}\renewcommand{\labelitemi}{$\Box$}%
\begin{itemize}}{\end{itemize}}
\pagestyle{threepartheadings}
\usepackage{url}
\usepackage{wrapfig}
% removing the standard hyperref to avoid the horrible boxes
%\usepackage{hyperref}
\usepackage[hidelinks]{hyperref}
% added in the dtklogos for the bibtex formatting
\usepackage{dtklogos}
%=========================
% One-inch margins everywhere
%=========================
\setlength{\topmargin}{0in}
\setlength{\textheight}{8.5in}
\setlength{\oddsidemargin}{0in}
\setlength{\evensidemargin}{0in}
\setlength{\textwidth}{6.5in}
%===============================
%===============================
% Macro for document title:
%===============================
\newcommand{\MYTITLE}[1]%
   {\begin{center}
     \begin{center}
     \bf
     CMPSC 580\\Topics and Research Methods in Computer Science\\
     Spring 2014
     \medskip
     \end{center}
     \bf
     #1
     \end{center}
}
%================================
% Macro for headings:
%================================
\newcommand{\MYHEADERS}[2]%
   {\lhead{#1}
    \rhead{#2}
    %\immediate\write16{}
    %\immediate\write16{DATE OF HANDOUT?}
    %\read16 to \dateofhandout
    \def \dateofhandout {January 14, 2014}
    \lfoot{\sc Handed out on \dateofhandout}
    %\immediate\write16{}
    %\immediate\write16{HANDOUT NUMBER?}
    %\read16 to\handoutnum
    \def \handoutnum {1}
    \rfoot{Handout \handoutnum}
   }

%================================
% Macro for bold italic:
%================================
\newcommand{\bit}[1]{{\textit{\textbf{#1}}}}

%=========================
% Non-zero paragraph skips.
%=========================
\setlength{\parskip}{1ex}

%=========================
% Create various environments:
%=========================
\newcommand{\PURPOSE}{\par\noindent\hspace{-.25in}{\bf Purpose:\ }}
\newcommand{\SUMMARY}{\par\noindent\hspace{-.25in}{\bf Summary:\ }}
\newcommand{\DETAILS}{\par\noindent\hspace{-.25in}{\bf Details:\ }}
\newcommand{\HANDIN}{\par\noindent\hspace{-.25in}{\bf Hand in:\ }}
\newcommand{\SUBHEAD}[1]{\bigskip\par\noindent\hspace{-.1in}{\sc #1}\\}
%\newenvironment{CHECKLIST}{\begin{itemize}}{\end{itemize}}

\usepackage[compact]{titlesec}

\begin{document}

\MYTITLE{Assignment 7: Research Paper Review\\Due Thursday, 29 January 2015}
\MYHEADERS{Assignment 7}{Due Thursday, 29 January 2015}

\subsection*{General Information}

For Module 2 you will read two research articles and use \LaTeX\ to write a one-page review of them.

\subsection*{Overview of the Papers}

  Automated unit test generation tools take as input a module under test and produce as output a test suite for that
  module. In the case of a tool called EvoSuite, the input is a Java class and the output is a JUnit test suite for that
  class. Although automated test data generation is an interesting idea, it raises important questions about the
  efficiency and effectiveness of the data generator itself. The first paper, ``A Large Scale Evaluation of Automated
  Unit Test Generation using EvoSuite'' presents one of the largest empirical studies to answer these types of
  questions. The second paper, ``Parameter Tuning for Search-Based Test-Data Generation Revisited: Support for Previous
  Results'' reports on a replication of previous experimentation with EvoSuite. While the first paper uses EvoSuite in a
  default configuration, the second attempts to determine whether or not modifying the parameters of the data generator
  will lead to higher quality test suites.

\subsection*{The Assignment}

  Both of the assigned papers describe automated test data generation techniques and then report on empirical studies of
  their effectiveness.  Once you are finished reading these papers, you should know what automatic test data generation
  is, how it works, and the circumstances in which it is applied.  Furthermore, you should have a comprehensive
  understanding of when automated test data generators do and do not work well.  Since they include graphs, tables of
  data, and statistical analyses, these papers will also enable you to develop a full-featured understanding of how to
  conduct and report on the results of experiments in the field of computer science.

  After you have finished reading the papers, use \LaTeX\ to write a one-page summary encompassing them both.  This does
  not require full comprehension of the material presented in the papers --- try to train your eyes to pick up on the
  ``big picture'' without getting lost in the extraneous details of the methodology, implementation of a technique, or
  the empirical results.

  Your summary, devoting about half a page for each paper, should respond to these questions:

\begin{itemize}
  \itemsep0in
  \item What is the purpose and the background of the research described in the paper? Why is the proposed research
    important and interesting?

  \item What does the related literature suggest about the presented research? How does it further justify the motivation
    behind this research?

  \item In broad terms, what are the methodologies and techniques described in the paper? 

  \item What do the results of the paper show? What are the implications of the results?

\end{itemize}

\end{document}
