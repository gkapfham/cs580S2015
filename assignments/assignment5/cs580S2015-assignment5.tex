%!TEX root=cs580S2015-assignment5.tex
% mainfile: cs580S2015-assignment5.tex
%!TEX root=cs580S2015-syllabus.tex
% mainfile: cs580S2015-syllabus.tex 

% CS 580 style
% Typical usage (all UPPERCASE items are optional):
%       \input 580pre
%       \begin{document}
%       \MYTITLE{Title of document, e.g., Lab 1\\Due ...}
%       \MYHEADERS{short title}{other running head, e.g., due date}
%       \PURPOSE{Description of purpose}
%       \SUMMARY{Very short overview of assignment}
%       \DETAILS{Detailed description}
%         \SUBHEAD{if needed} ...
%         \SUBHEAD{if needed} ...
%          ...
%       \HANDIN{What to hand in and how}
%       \begin{checklist}
%       \item ...
%       \end{checklist}
% There is no need to include a "\documentstyle."
% However, there should be an "\end{document}."
%
%===========================================================

\documentclass[11pt,twoside,titlepage]{article}

\usepackage{threeparttop}
\usepackage{graphicx}
\usepackage{latexsym}
\usepackage{color}
\usepackage{listings}
\usepackage{fancyvrb}
%\usepackage{pgf,pgfarrows,pgfnodes,pgfautomata,pgfheaps,pgfshade}
\usepackage{tikz}
\usepackage[normalem]{ulem}
\tikzset{
    %Define standard arrow tip
%    >=stealth',
    %Define style for boxes
    oval/.style={
           rectangle,
           rounded corners,
           draw=black, very thick,
           text width=6.5em,
           minimum height=2em,
           text centered},
    % Define arrow style
    arr/.style={
           ->,
           thick,
           shorten <=2pt,
           shorten >=2pt,}
}
\usepackage[noend]{algorithmic}
\usepackage[noend]{algorithm}
\newcommand{\bfor}{{\bf for\ }}
\newcommand{\bthen}{{\bf then\ }}
\newcommand{\bwhile}{{\bf while\ }}
\newcommand{\btrue}{{\bf true\ }}
\newcommand{\bfalse}{{\bf false\ }}
\newcommand{\bto}{{\bf to\ }}
\newcommand{\bdo}{{\bf do\ }}
\newcommand{\bif}{{\bf if\ }}
\newcommand{\belse}{{\bf else\ }}
\newcommand{\band}{{\bf and\ }}
\newcommand{\breturn}{{\bf return\ }}
\newcommand{\mod}{{\rm mod}}
\renewcommand{\algorithmiccomment}[1]{$\rhd$ #1}
\newenvironment{checklist}{\par\noindent\hspace{-.25in}{\bf Checklist:}\renewcommand{\labelitemi}{$\Box$}%
\begin{itemize}}{\end{itemize}}
\pagestyle{threepartheadings}
\usepackage{url}
\usepackage{wrapfig}
% removing the standard hyperref to avoid the horrible boxes
%\usepackage{hyperref}
\usepackage[hidelinks]{hyperref}
% added in the dtklogos for the bibtex formatting
\usepackage{dtklogos}
%=========================
% One-inch margins everywhere
%=========================
\setlength{\topmargin}{0in}
\setlength{\textheight}{8.5in}
\setlength{\oddsidemargin}{0in}
\setlength{\evensidemargin}{0in}
\setlength{\textwidth}{6.5in}
%===============================
%===============================
% Macro for document title:
%===============================
\newcommand{\MYTITLE}[1]%
   {\begin{center}
     \begin{center}
     \bf
     CMPSC 580\\Topics and Research Methods in Computer Science\\
     Spring 2014
     \medskip
     \end{center}
     \bf
     #1
     \end{center}
}
%================================
% Macro for headings:
%================================
\newcommand{\MYHEADERS}[2]%
   {\lhead{#1}
    \rhead{#2}
    %\immediate\write16{}
    %\immediate\write16{DATE OF HANDOUT?}
    %\read16 to \dateofhandout
    \def \dateofhandout {January 14, 2014}
    \lfoot{\sc Handed out on \dateofhandout}
    %\immediate\write16{}
    %\immediate\write16{HANDOUT NUMBER?}
    %\read16 to\handoutnum
    \def \handoutnum {1}
    \rfoot{Handout \handoutnum}
   }

%================================
% Macro for bold italic:
%================================
\newcommand{\bit}[1]{{\textit{\textbf{#1}}}}

%=========================
% Non-zero paragraph skips.
%=========================
\setlength{\parskip}{1ex}

%=========================
% Create various environments:
%=========================
\newcommand{\PURPOSE}{\par\noindent\hspace{-.25in}{\bf Purpose:\ }}
\newcommand{\SUMMARY}{\par\noindent\hspace{-.25in}{\bf Summary:\ }}
\newcommand{\DETAILS}{\par\noindent\hspace{-.25in}{\bf Details:\ }}
\newcommand{\HANDIN}{\par\noindent\hspace{-.25in}{\bf Hand in:\ }}
\newcommand{\SUBHEAD}[1]{\bigskip\par\noindent\hspace{-.1in}{\sc #1}\\}
%\newenvironment{CHECKLIST}{\begin{itemize}}{\end{itemize}}

\usepackage[compact]{titlesec}

\begin{document}

\MYTITLE{Assignment 5\\Managing Technical Writing with Git\\Due 23
January 2015}
\MYHEADERS{Managing Technical Writing with Git and Vim}{Due 23 January 2015}

\section*{Introduction}

  Researchers and developers in the field of computer science normally use a version control system to manage most of the
  artifacts produced during many of the phases associated with their work. In this course, we will always use the Git
  distributed version control system to manage the files associated with our assignments.  In this assignment, you will
  learn how to use the Bitbucket service for managing Git repositories and the {\tt git} command-line tool in the Ubuntu
  Linux operating system.

% Beyond using a version control system, researchers and developers in computer science often use an integrated
% development environment (IDE) for the creation of papers that contain paragraphs of text, formal statements of
% algorithms, mathematical equations, technical diagrams, symbolic references, lengthy bibliographies, and other important
% items.  In this course, we will always use Vim and GVim when we complete technical writing assignments involving, for
% instance, the creation of a proposal or presentation slides. In this assignment, you will learn the basics associated
% with the use of Vim to write a technical document stored in a Git repository.

\section*{Configuring Git and Bitbucket}

During this assignment and subsequent assignments, we will securely communicate with the Bitbucket.org servers that will
host our all of our projects.  In this assignment, we will perform all of the steps to configure the accounts
on the departmental servers and the Bitbucket service.  Throughout the assignment, you should refer to the following Web
site for additional information: \url{https://confluence.atlassian.com/display/BITBUCKET/Bitbucket+101}.

Throughout the completion of this assignment, please be sure to keep a record of all of the steps that you took to
configure Git and Bitbucket.  You should also practice the steps multiple times to ensure that you can easily use Git
throughout the remainder of the academic semester.

Please follow the below steps to configure and begin to use the Git service provided by Bitbucket:

\begin{enumerate}

  \item If you have never done so before, you must use the {\tt ssh-keygen} program to create secure-shell keys that
    you can use to support your communication with the Bitbucket servers.  Type {\tt man ssh-keygen} and talk with
    one of the course instructors to learn more about how to use this program.  What files does {\tt ssh-keygen} produce?
    Where does this program store these files?

  \item If you do not already have a Bitbucket account, please go to the Bitbucket Web site and create one ---
    make sure that you use your {\tt allegheny.edu} email address so that you can create an unlimited number of free
    Bitbucket repositories.

  \item Now, you need to test to see if you can authenticate with the Bitbucket servers. First, show one of the course
    instructors that you have correctly configured your Bitbucket account.  Now, ask this instructor to share the
    course's Git repository with you.  Open a terminal window on your workstation and change into the directory where
    you will store your files for this course.  For instance, you might make a {\tt cs580S2015/} directory that will
    contain the Git repository that we will always use to share files with you.  Next, please type the following
    command: {\tt git clone git@bitbucket.org:gkapfham/cs580s2015-share.git}.  If everything worked correctly, you
    should be able to download some files that you can use for this assignment. Please resolve any problems that you
    encountered by first reviewing the Bitbucket documentation and then asking a course instructor for additional
    assistance.

\end{enumerate}

\section*{Creating a New Repository}

  % Now, go into the {\tt
  % assignments/survey-of-interests-assignment1} directory of the {\tt cs580s2015-share} repository and copy the three
% files to a suitable directory inside of a directory in your repository. Once the files are in your own Git repository,
% please use the {\tt git add} and {\tt git commit} commands to add them correctly. 

Now that you have learned how to clone an existing Git repository, you should create a new repository in the {\tt
cs580S2015/} directory that you previously created.  First, create a new directory called {\tt cs580S2015-<your user
name>}. Then, change into this directory and type the command {\tt git init .}. Then, you need to start adding some
files into your repository. For instance, if you have already types in your survey of interests, you could add that to
your repository using the command {\tt git add} and {\tt git commit}.  Next, you should use the Bitbucket Web site to
create a repository that has the same name as the local directory and local repository.  You must follow Bitbucket's
instructions to push the code and tags in your local repository to the remote one.

Students who would like to learn more about Git can consult Web sites like \url{http://try.github.io/} and
\url{http://gitimmersion.com/}.  At minimum, you should ensure that you fully understand how to use the following Git
commands in the terminal window:

\begin{enumerate}

        \item {\tt git init}

        \item {\tt git status}

        \item {\tt git add}

        \item {\tt git commit}

        \item {\tt git push}

        \item {\tt git pull}

\end{enumerate}

In summary, the main point of this assignment is to review how to create and use Git and Bitbucket. We will leverage
these tools throughout the remainder of the semester.  You should make sure that you are able to both access code that I
share with you and share code with me through a Git repository. Please see the course instructor if you cannot complete
these steps.


\end{document}
