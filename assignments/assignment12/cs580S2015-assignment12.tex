%!TEX root=cs580S2015-assignment12.tex
% mainfile: cs580S2015-assignment12.tex 
\input{580pre.tex}
\usepackage{ulem}
\usepackage[compact]{titlesec}

\begin{document}

\MYTITLE{Assignment 12\\Proposal Writing Workshop, Part 2\\
Draft due Friday, 13 February}
\MYHEADERS{Assignment 12}{Due Fri., 13 Feb.}

Today we continue the team-editing approach to developing a proposal.  You will persist in working with your partners to
ensure that you have a strong draft of a proposal, that connects to the theme of this module, and is ready to turn in by
Friday of this week.

\subsection*{Quick Review of \BibTeX}

The first few minutes of Thursday's class will be given over to a quick review of \BibTeX. Make sure that, by the end of
this module, you are comfortable with finding bibliographic entries in the ACM Digital Library, adding these entries to
your \BibTeX\ file, and then citing them in your proposal.

\subsection*{Discussing, Developing, and Revising Your Proposal}

For the remainder of the class, continue to work with your partners to add content to your proposal. You have each
received feedback, from you team members, on your outline---use the suggestions given by them or try to come up with
similar suggestions based on the same reasoning.

You should particularly concentrate on producing a good example, demonstration, or other motivating device for your
proposed topic. Instead of giving your personal motivation for investigating your chosen area, the proposal should
include professional, technical, and mathematical motivations for pursuing the topic.  You should also try to locate
good references on both your topic and the broader topics that connect to it (e.g., general references on ``software
engineering'' or ``software testing'' or ``program debugging'' or ``evolutionary computation'').

By Friday, you should submit a rough draft of your proposal. This draft should contain a title, a preliminary abstract,
an outline of all the sections in your proposal, a bibliography with at least five references, appropriate citations to these
chosen references in the main text, and paragraphs of draft text in all of the main sections. Whenever possible, your
draft should contain concrete examples, the statement of a hypothesis and evaluation metrics, a description of
your method of approach, a plan for completing your work, and any other relevant components of a proposal.

\end{document}
