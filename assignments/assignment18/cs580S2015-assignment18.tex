%!TEX root=cs580S2015-assignment18.tex
% mainfile: cs580S2015-assignment18.tex

%!TEX root=cs580S2015-syllabus.tex
% mainfile: cs580S2015-syllabus.tex 

% CS 580 style
% Typical usage (all UPPERCASE items are optional):
%       \input 580pre
%       \begin{document}
%       \MYTITLE{Title of document, e.g., Lab 1\\Due ...}
%       \MYHEADERS{short title}{other running head, e.g., due date}
%       \PURPOSE{Description of purpose}
%       \SUMMARY{Very short overview of assignment}
%       \DETAILS{Detailed description}
%         \SUBHEAD{if needed} ...
%         \SUBHEAD{if needed} ...
%          ...
%       \HANDIN{What to hand in and how}
%       \begin{checklist}
%       \item ...
%       \end{checklist}
% There is no need to include a "\documentstyle."
% However, there should be an "\end{document}."
%
%===========================================================

\documentclass[11pt,twoside,titlepage]{article}

\usepackage{threeparttop}
\usepackage{graphicx}
\usepackage{latexsym}
\usepackage{color}
\usepackage{listings}
\usepackage{fancyvrb}
%\usepackage{pgf,pgfarrows,pgfnodes,pgfautomata,pgfheaps,pgfshade}
\usepackage{tikz}
\usepackage[normalem]{ulem}
\tikzset{
    %Define standard arrow tip
%    >=stealth',
    %Define style for boxes
    oval/.style={
           rectangle,
           rounded corners,
           draw=black, very thick,
           text width=6.5em,
           minimum height=2em,
           text centered},
    % Define arrow style
    arr/.style={
           ->,
           thick,
           shorten <=2pt,
           shorten >=2pt,}
}
\usepackage[noend]{algorithmic}
\usepackage[noend]{algorithm}
\newcommand{\bfor}{{\bf for\ }}
\newcommand{\bthen}{{\bf then\ }}
\newcommand{\bwhile}{{\bf while\ }}
\newcommand{\btrue}{{\bf true\ }}
\newcommand{\bfalse}{{\bf false\ }}
\newcommand{\bto}{{\bf to\ }}
\newcommand{\bdo}{{\bf do\ }}
\newcommand{\bif}{{\bf if\ }}
\newcommand{\belse}{{\bf else\ }}
\newcommand{\band}{{\bf and\ }}
\newcommand{\breturn}{{\bf return\ }}
\newcommand{\mod}{{\rm mod}}
\renewcommand{\algorithmiccomment}[1]{$\rhd$ #1}
\newenvironment{checklist}{\par\noindent\hspace{-.25in}{\bf Checklist:}\renewcommand{\labelitemi}{$\Box$}%
\begin{itemize}}{\end{itemize}}
\pagestyle{threepartheadings}
\usepackage{url}
\usepackage{wrapfig}
% removing the standard hyperref to avoid the horrible boxes
%\usepackage{hyperref}
\usepackage[hidelinks]{hyperref}
% added in the dtklogos for the bibtex formatting
\usepackage{dtklogos}
%=========================
% One-inch margins everywhere
%=========================
\setlength{\topmargin}{0in}
\setlength{\textheight}{8.5in}
\setlength{\oddsidemargin}{0in}
\setlength{\evensidemargin}{0in}
\setlength{\textwidth}{6.5in}
%===============================
%===============================
% Macro for document title:
%===============================
\newcommand{\MYTITLE}[1]%
   {\begin{center}
     \begin{center}
     \bf
     CMPSC 580\\Topics and Research Methods in Computer Science\\
     Spring 2014
     \medskip
     \end{center}
     \bf
     #1
     \end{center}
}
%================================
% Macro for headings:
%================================
\newcommand{\MYHEADERS}[2]%
   {\lhead{#1}
    \rhead{#2}
    %\immediate\write16{}
    %\immediate\write16{DATE OF HANDOUT?}
    %\read16 to \dateofhandout
    \def \dateofhandout {January 14, 2014}
    \lfoot{\sc Handed out on \dateofhandout}
    %\immediate\write16{}
    %\immediate\write16{HANDOUT NUMBER?}
    %\read16 to\handoutnum
    \def \handoutnum {1}
    \rfoot{Handout \handoutnum}
   }

%================================
% Macro for bold italic:
%================================
\newcommand{\bit}[1]{{\textit{\textbf{#1}}}}

%=========================
% Non-zero paragraph skips.
%=========================
\setlength{\parskip}{1ex}

%=========================
% Create various environments:
%=========================
\newcommand{\PURPOSE}{\par\noindent\hspace{-.25in}{\bf Purpose:\ }}
\newcommand{\SUMMARY}{\par\noindent\hspace{-.25in}{\bf Summary:\ }}
\newcommand{\DETAILS}{\par\noindent\hspace{-.25in}{\bf Details:\ }}
\newcommand{\HANDIN}{\par\noindent\hspace{-.25in}{\bf Hand in:\ }}
\newcommand{\SUBHEAD}[1]{\bigskip\par\noindent\hspace{-.1in}{\sc #1}\\}
%\newenvironment{CHECKLIST}{\begin{itemize}}{\end{itemize}}

\usepackage[compact]{titlesec}

\begin{document}

\MYTITLE{Assignment 18\\Proposal Writing Workshop, Parts 1 and 2\\
Outline due at the end of class Friday, 13 March}
\MYHEADERS{Assignment 18}{Due Friday, 13 Mar.}

During class on Tuesday and Thursday of this week, you will be working with a partner to produce an outline and first
draft of your module three proposal. In addition to preparing a draft of your proposal, you should take this time to
work with the course instructor to enhance your knowledge of technical document preparation in \LaTeX.

\subsection*{Reminder About How Peer Editing Works}

You and your partner will go through a cycle of ``discussion,'' ``development,'' and ``review''  modes.  During
discussion mode, you will present to each other a small component of the proposal and talk about it---``bounce ideas''
off each other, ask questions, and take notes on one anothers' comments and concerns.  This will be followed by a period
of writing (development) when you will each try to create a small portion of formal text that solidifies some of the
ideas you have discussed.  Following this will be a period of review when you read over each others' writing and comment
on it.

Each partner should have a fresh copy of the senior thesis proposal template
(there is a copy in the \url{cs580s2015-share/proposal-template} repository
folder).

\subsection*{Tuesday---First Round}

\noindent{\bf Discuss:} Share your idea for a senior project with your editing
partner and listen carefully to your partner's idea. If you are unclear about
the nature of the project, try to ask strategic questions about it that will
elicit more specifics. Examples of good questions (you can think of more!):

\begin{itemize}
\item
What would be a good title that summarizes the purpose and goals of the project?
\item
What will be the main ``deliverable'' of the thesis---a computer program or
a mobile app? an
experiment? an empirical study? a set of recommendations? a
public Web site? an analysis of an existing program, process, or system?
a piece of hardware? a case study? \ldots
\item
How will the correctness, validity, effectiveness, efficiency, etc., of the
project be
evaluated and measured---through experiments? using established benchmarks?
through human subject testing? by mathematical analysis? \ldots (The word
``metrics'' is often used in conjunction with this---what
metrics will be used?)
\item
Is there a very simple example of the problem or the form
of the proposed deliverable?
\item
Is there a body of professional literature (e.g., books, journal articles)
dealing with the topic?
\end{itemize}

\noindent{\bf Develop:}
Following this discussion (no more than 15 minutes initially), each writer
should spend 10 or 15 minutes attempting to write a brief introductory
paragraph describing the project.
Please eliminate filler text, diagrams, and tables from the
template so that only your new text appears.

\noindent{\bf Review:} All of the team members should review one anothers'
description, making suggestions about both the technical content and the writing style.

\subsection*{Tuesday---Second Round}

\noindent{\bf Discuss:} Each student should try to outline the major
components of a proposal on his or her chosen topic and should listen to
and ask questions about the partner's outline. This should be a
specific, rather than a generic, outline. For instance, ``Review of
Literature'' is too general---instead, mention specific publications
that should be reviewed. Examples of questions:

\begin{itemize}
\item
What is the chief motivation for studying the proposed topic? Can this
motivation be conveyed to the reader using statistics, anecdotes, hypothetical
examples, recent news articles, or other means?
\item
Who will benefit from the proposed research---Web users? programmers?
businesses? artists? environmental researchers? students in programming classes?
\ldots
\item
Is there a concept that has not been discussed in the course that is central
to understanding the proposal? If so, should it have its own section in the
proposal?
\item
Is there a backup plan in case the proposed work turns out not to be feasible?
\end{itemize}

\noindent{\bf Develop:} Try to organize the section headings in the proposal template
according to the outline you and your partner have discussed, filling in text
wherever possible. Remember that \LaTeX\ allows subsections, so several
levels of outlining are possible.

\noindent{\bf Review:} Critique each others' outlines, then print out your
skeleton document and hand it in at the end of the class. Remember to place a
copy in your personal {\tt git} repository and make sure that you have shared it
with all members of the faculty.

\subsection*{Preparation For Thursday}

Between Tuesday and Thursday, try to develop some of the sections you outlined in class on Tuesday. This would be an
excellent time to track down some references and give them a first reading, and begin to format them \BibTeX-style. It
would also be a great time to expand the introduction of your proposal based on the questions and suggestions you
received from your partner.

\subsection*{Final Submission on Friday}

You should particularly concentrate on producing a good example, demonstration, or other motivating device for your
proposed topic. Instead of giving your personal motivation for investigating your chosen area, the proposal should
include professional, technical, and mathematical motivations for pursuing the topic.  You should also try to locate
good references on both your topic and the broader topics that connect to it (e.g., general references on ``software
engineering'' or ``software testing'' or ``program debugging'' or ``evolutionary computation'').

By Friday, you should submit a rough draft of your proposal. This draft should contain a title, a preliminary abstract,
an outline of all the sections in your proposal, a bibliography with at least five references, appropriate citations to these
chosen references in the main text, and paragraphs of draft text in all of the main sections. Whenever possible, your
draft should contain concrete examples, the statement of a hypothesis and evaluation metrics, a description of
your method of approach, a plan for completing your work, and any other relevant components of a proposal.

\end{document}
