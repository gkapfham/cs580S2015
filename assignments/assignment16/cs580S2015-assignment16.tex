%!TEX root=cs580S2015-syllabus.tex
% mainfile: cs580S2015-syllabus.tex 

% CS 580 style
% Typical usage (all UPPERCASE items are optional):
%       \input 580pre
%       \begin{document}
%       \MYTITLE{Title of document, e.g., Lab 1\\Due ...}
%       \MYHEADERS{short title}{other running head, e.g., due date}
%       \PURPOSE{Description of purpose}
%       \SUMMARY{Very short overview of assignment}
%       \DETAILS{Detailed description}
%         \SUBHEAD{if needed} ...
%         \SUBHEAD{if needed} ...
%          ...
%       \HANDIN{What to hand in and how}
%       \begin{checklist}
%       \item ...
%       \end{checklist}
% There is no need to include a "\documentstyle."
% However, there should be an "\end{document}."
%
%===========================================================

\documentclass[11pt,twoside,titlepage]{article}

\usepackage{threeparttop}
\usepackage{graphicx}
\usepackage{latexsym}
\usepackage{color}
\usepackage{listings}
\usepackage{fancyvrb}
%\usepackage{pgf,pgfarrows,pgfnodes,pgfautomata,pgfheaps,pgfshade}
\usepackage{tikz}
\usepackage[normalem]{ulem}
\tikzset{
    %Define standard arrow tip
%    >=stealth',
    %Define style for boxes
    oval/.style={
           rectangle,
           rounded corners,
           draw=black, very thick,
           text width=6.5em,
           minimum height=2em,
           text centered},
    % Define arrow style
    arr/.style={
           ->,
           thick,
           shorten <=2pt,
           shorten >=2pt,}
}
\usepackage[noend]{algorithmic}
\usepackage[noend]{algorithm}
\newcommand{\bfor}{{\bf for\ }}
\newcommand{\bthen}{{\bf then\ }}
\newcommand{\bwhile}{{\bf while\ }}
\newcommand{\btrue}{{\bf true\ }}
\newcommand{\bfalse}{{\bf false\ }}
\newcommand{\bto}{{\bf to\ }}
\newcommand{\bdo}{{\bf do\ }}
\newcommand{\bif}{{\bf if\ }}
\newcommand{\belse}{{\bf else\ }}
\newcommand{\band}{{\bf and\ }}
\newcommand{\breturn}{{\bf return\ }}
\newcommand{\mod}{{\rm mod}}
\renewcommand{\algorithmiccomment}[1]{$\rhd$ #1}
\newenvironment{checklist}{\par\noindent\hspace{-.25in}{\bf Checklist:}\renewcommand{\labelitemi}{$\Box$}%
\begin{itemize}}{\end{itemize}}
\pagestyle{threepartheadings}
\usepackage{url}
\usepackage{wrapfig}
% removing the standard hyperref to avoid the horrible boxes
%\usepackage{hyperref}
\usepackage[hidelinks]{hyperref}
% added in the dtklogos for the bibtex formatting
\usepackage{dtklogos}
%=========================
% One-inch margins everywhere
%=========================
\setlength{\topmargin}{0in}
\setlength{\textheight}{8.5in}
\setlength{\oddsidemargin}{0in}
\setlength{\evensidemargin}{0in}
\setlength{\textwidth}{6.5in}
%===============================
%===============================
% Macro for document title:
%===============================
\newcommand{\MYTITLE}[1]%
   {\begin{center}
     \begin{center}
     \bf
     CMPSC 580\\Topics and Research Methods in Computer Science\\
     Spring 2014
     \medskip
     \end{center}
     \bf
     #1
     \end{center}
}
%================================
% Macro for headings:
%================================
\newcommand{\MYHEADERS}[2]%
   {\lhead{#1}
    \rhead{#2}
    %\immediate\write16{}
    %\immediate\write16{DATE OF HANDOUT?}
    %\read16 to \dateofhandout
    \def \dateofhandout {January 14, 2014}
    \lfoot{\sc Handed out on \dateofhandout}
    %\immediate\write16{}
    %\immediate\write16{HANDOUT NUMBER?}
    %\read16 to\handoutnum
    \def \handoutnum {1}
    \rfoot{Handout \handoutnum}
   }

%================================
% Macro for bold italic:
%================================
\newcommand{\bit}[1]{{\textit{\textbf{#1}}}}

%=========================
% Non-zero paragraph skips.
%=========================
\setlength{\parskip}{1ex}

%=========================
% Create various environments:
%=========================
\newcommand{\PURPOSE}{\par\noindent\hspace{-.25in}{\bf Purpose:\ }}
\newcommand{\SUMMARY}{\par\noindent\hspace{-.25in}{\bf Summary:\ }}
\newcommand{\DETAILS}{\par\noindent\hspace{-.25in}{\bf Details:\ }}
\newcommand{\HANDIN}{\par\noindent\hspace{-.25in}{\bf Hand in:\ }}
\newcommand{\SUBHEAD}[1]{\bigskip\par\noindent\hspace{-.1in}{\sc #1}\\}
%\newenvironment{CHECKLIST}{\begin{itemize}}{\end{itemize}}

\usepackage[compact]{titlesec}

\begin{document}

\MYTITLE{Assignment 16\\{\em Writing for Computer Science} Pr\'{e}cis\\
Due February 27, 2014}
\MYHEADERS{Assignment 16}{Due Fri., Feb. 27}


\subsection*{Write a Pr\'{e}cis}

Using \LaTeX, create a one-page pr\'{e}cis of chapters 3 and 6 in Zobel's {\em Writing for Computer Science.} Write in a
formal scholarly style. Review what you have learned in your FS and other courses about formal writing---write simply
and objectively, paying attention to organization, grammar, spelling, and style. You do not need to provide citations,
footnotes, or a bibliography. Overall, this assignment is similar to assignment 8 where you wrote a pr\'{e}cis of
chapters 1 and 2.

% Write in a formal scholarly style. Your writing should simple and objective, and it should have clear organization, use
% proper grammar, have no spelling mistakes, and use a proper writing style. You do not need to provide citations,
% footnotes, or a bibliography.

\subsection*{Additional Advice}

The Junior Seminar is meant to simulate a real research group.  A member of a research group should not feel inhibited
about approaching another member of the group (or, for that matter, a faculty member) for advice, or to discuss an
interpretation of a reading, or to ask for critical feedback on a document, project, or presentation.  Similarly, group
members should always be willing to act as a sounding board or to provide feedback to others.

% \begin{quote}
% \color{blue}
% You may (and, in fact, you are \textit{\textbf{encouraged}} to) discuss the chapters among
% yourselves and try to identify the most crucial points.
% You may (and, in fact, you are  \textit{\textbf{encouraged}} to) share
% drafts of this and future writing assignments with one another for review
% and constructive criticism.
% \end{quote}

In summary, you should collaborate with other students in the research seminar:

\begin{itemize}

\item You are {encouraged} to discuss the assigned chapters among yourselves and try to identify the
  most crucial points developed by the author.

\item You are  encouraged to share drafts of this assignment with one another for review and constructive criticism,
  including writing suggestions, finding typos, grammatical errors, etc.

\item If you are approached by a classmate, I expect you to agree to review that person's writing.  Exchange
  papers and learn how to evaluate others' work---it will make you a better writer!

\end{itemize}

It is possible to work together and still produce a document that is your own. The spirit of the Honor Code must still
predominate, but you should now possess the maturity to be able to distinguish when you are ``growing your ideas'' in
the fertile soil of shared discussion and criticism and when you are ``taking somebody else's work and presenting it as
your own.'' Do good work. Help one another. Please see the course instructor if you have questions about this policy.

\end{document}
