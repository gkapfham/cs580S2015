\input{580pre.tex}
\usepackage[compact]{titlesec}

\begin{document}

\MYTITLE{Assignment 16\\{\em Writing for Computer Science} Pr\'{e}cis\\
Due February 27, 2014}
\MYHEADERS{Assignment 16}{Due Fri., Feb. 27}


\subsection*{Write a Pr\'{e}cis} 

Using \LaTeX, create a one-page pr\'{e}cis of chapters 2 and 6 in Zobel's {\em Writing for Computer Science.} Write in a
formal scholarly style. Review what you have learned in your FS and other courses about formal writing---write simply
and objectively, paying attention to organization, grammar, spelling, and style. You do not need to provide citations,
footnotes, or a bibliography. Overall, this assignment is similar to assignment 7 where you wrote pr\'{e}cis of chapters
1 and 10. 

% Write in a formal scholarly style. Your writing should simple and objective, and it should have clear organization, use
% proper grammar, have no spelling mistakes, and use a proper writing style. You do not need to provide citations,
% footnotes, or a bibliography.

\subsection*{Additional Advice}

The Junior Seminar is meant to simulate a real research group.  A member of a research group should not feel inhibited
about approaching another member of the group (or, for that matter, a faculty member) for advice, or to discuss an
interpretation of a reading, or to ask for critical feedback on a document, project, or presentation.  Similarly, group
members should always be willing to act as a sounding board or to provide feedback to others.  

% \begin{quote}
% \color{blue}
% You may (and, in fact, you are \textit{\textbf{encouraged}} to) discuss the chapters among
% yourselves and try to identify the most crucial points.
% You may (and, in fact, you are  \textit{\textbf{encouraged}} to) share
% drafts of this and future writing assignments with one another for review
% and constructive criticism. 
% \end{quote}

You should collaborate with other students in the research seminar:

\begin{itemize}

\item You are \textit{\textbf{encouraged}} to discuss the assigned chapters among yourselves and try to identify the
  most crucial points developed by the author.

\item You are  \textit{\textbf{encouraged}} to share drafts of this assignment with one another for review and
  constructive criticism, including writing suggestions, finding typos, grammatical errors, etc.

\item If you are approached by a classmate, I expect you to agree to review that person's writing.  Exchange
  papers and learn how to evaluate the others' work---it will make you a better writer!

\end{itemize}

It is possible to do this and still produce a document that is your own. The spirit of the Honor Code must still
predominate, but you should now possess the maturity to be able to distinguish when you are ``growing your ideas'' in
the fertile soil of shared discussion and criticism and when you are ``taking somebody else's work and presenting it as
your own.'' Do good work. Help one another. Please see the course instructor if you have questions about this important policy.

\end{document}
