%!TEX root=cs580S2015-assignment6.tex
% mainfile: cs580S2015-assignment6.tex 
%!TEX root=cs580S2015-syllabus.tex
% mainfile: cs580S2015-syllabus.tex 

% CS 580 style
% Typical usage (all UPPERCASE items are optional):
%       \input 580pre
%       \begin{document}
%       \MYTITLE{Title of document, e.g., Lab 1\\Due ...}
%       \MYHEADERS{short title}{other running head, e.g., due date}
%       \PURPOSE{Description of purpose}
%       \SUMMARY{Very short overview of assignment}
%       \DETAILS{Detailed description}
%         \SUBHEAD{if needed} ...
%         \SUBHEAD{if needed} ...
%          ...
%       \HANDIN{What to hand in and how}
%       \begin{checklist}
%       \item ...
%       \end{checklist}
% There is no need to include a "\documentstyle."
% However, there should be an "\end{document}."
%
%===========================================================

\documentclass[11pt,twoside,titlepage]{article}

\usepackage{threeparttop}
\usepackage{graphicx}
\usepackage{latexsym}
\usepackage{color}
\usepackage{listings}
\usepackage{fancyvrb}
%\usepackage{pgf,pgfarrows,pgfnodes,pgfautomata,pgfheaps,pgfshade}
\usepackage{tikz}
\usepackage[normalem]{ulem}
\tikzset{
    %Define standard arrow tip
%    >=stealth',
    %Define style for boxes
    oval/.style={
           rectangle,
           rounded corners,
           draw=black, very thick,
           text width=6.5em,
           minimum height=2em,
           text centered},
    % Define arrow style
    arr/.style={
           ->,
           thick,
           shorten <=2pt,
           shorten >=2pt,}
}
\usepackage[noend]{algorithmic}
\usepackage[noend]{algorithm}
\newcommand{\bfor}{{\bf for\ }}
\newcommand{\bthen}{{\bf then\ }}
\newcommand{\bwhile}{{\bf while\ }}
\newcommand{\btrue}{{\bf true\ }}
\newcommand{\bfalse}{{\bf false\ }}
\newcommand{\bto}{{\bf to\ }}
\newcommand{\bdo}{{\bf do\ }}
\newcommand{\bif}{{\bf if\ }}
\newcommand{\belse}{{\bf else\ }}
\newcommand{\band}{{\bf and\ }}
\newcommand{\breturn}{{\bf return\ }}
\newcommand{\mod}{{\rm mod}}
\renewcommand{\algorithmiccomment}[1]{$\rhd$ #1}
\newenvironment{checklist}{\par\noindent\hspace{-.25in}{\bf Checklist:}\renewcommand{\labelitemi}{$\Box$}%
\begin{itemize}}{\end{itemize}}
\pagestyle{threepartheadings}
\usepackage{url}
\usepackage{wrapfig}
% removing the standard hyperref to avoid the horrible boxes
%\usepackage{hyperref}
\usepackage[hidelinks]{hyperref}
% added in the dtklogos for the bibtex formatting
\usepackage{dtklogos}
%=========================
% One-inch margins everywhere
%=========================
\setlength{\topmargin}{0in}
\setlength{\textheight}{8.5in}
\setlength{\oddsidemargin}{0in}
\setlength{\evensidemargin}{0in}
\setlength{\textwidth}{6.5in}
%===============================
%===============================
% Macro for document title:
%===============================
\newcommand{\MYTITLE}[1]%
   {\begin{center}
     \begin{center}
     \bf
     CMPSC 580\\Topics and Research Methods in Computer Science\\
     Spring 2014
     \medskip
     \end{center}
     \bf
     #1
     \end{center}
}
%================================
% Macro for headings:
%================================
\newcommand{\MYHEADERS}[2]%
   {\lhead{#1}
    \rhead{#2}
    %\immediate\write16{}
    %\immediate\write16{DATE OF HANDOUT?}
    %\read16 to \dateofhandout
    \def \dateofhandout {January 14, 2014}
    \lfoot{\sc Handed out on \dateofhandout}
    %\immediate\write16{}
    %\immediate\write16{HANDOUT NUMBER?}
    %\read16 to\handoutnum
    \def \handoutnum {1}
    \rfoot{Handout \handoutnum}
   }

%================================
% Macro for bold italic:
%================================
\newcommand{\bit}[1]{{\textit{\textbf{#1}}}}

%=========================
% Non-zero paragraph skips.
%=========================
\setlength{\parskip}{1ex}

%=========================
% Create various environments:
%=========================
\newcommand{\PURPOSE}{\par\noindent\hspace{-.25in}{\bf Purpose:\ }}
\newcommand{\SUMMARY}{\par\noindent\hspace{-.25in}{\bf Summary:\ }}
\newcommand{\DETAILS}{\par\noindent\hspace{-.25in}{\bf Details:\ }}
\newcommand{\HANDIN}{\par\noindent\hspace{-.25in}{\bf Hand in:\ }}
\newcommand{\SUBHEAD}[1]{\bigskip\par\noindent\hspace{-.1in}{\sc #1}\\}
%\newenvironment{CHECKLIST}{\begin{itemize}}{\end{itemize}}

\usepackage[compact]{titlesec}

\begin{document}

\MYTITLE{Assignment 6\\Writing and Presenting with \LaTeX\ and Beamer\\
Due 30 January 2015}
\MYHEADERS{Assignment 6}{Due Fri., 30 Jan.}

\subsection*{Introduction to \LaTeX}
\LaTeX\ is a typesetting system used by computer scientists and mathematicians
the world over. Designed by Leslie Lamport (currently at Microsoft Research),
it is actually a set of procedures that build upon an earlier typesetting
system, \TeX, designed by Donald Knuth (retired from Stanford University).
You are strongly encouraged to read about Leslie Lamport and Donald Knuth 
online---they are pioneers in computer science research and are 
names you should know.

Unlike WYSIWYG word processors such as Microsoft Word or Open Office, \LaTeX\ 
uses a set of predefined commands, environments, special characters, etc.
which are assembled by the user into a description of a document. For
example, here is the \LaTeX\ source code and the typeset output
for a sample paragraph:
\begin{center}
\begin{tabular}{p{3in}p{3in}}
\multicolumn{1}{c}{\bf \LaTeX\ Source:}&
\multicolumn{1}{c}{\bf Output:}\\\hline
\begin{minipage}{3in}
\small
\begin{verbatim}
To determine the distance $d$ 
of a point $(x_0,y_0)$ from a 
second point $(x_1,y_1)$, we 
may use the \emph{distance
formula}:
\[d=\sqrt{(x_1-x_0)^2+(y_1-y_0)^2}\]
This generalizes to higher dimensions.
\end{verbatim}
\end{minipage}
&
\begin{minipage}{3in}
To determine the distance $d$ 
of a point $(x_0,y_0)$ from a 
second point $(x_1,y_1)$, we 
may use the \emph{distance
formula}:
\[d=\sqrt{(x_1-x_0)^2+(y_1-y_0)^2}\]
This generalizes to higher dimensions.
\end{minipage}
\end{tabular}
\end{center}

There are a number of useful software tools for working with \LaTeX,
including the {\tt vim} text editor. Get to know them---they are great
time-savers.  There is another program that we will be using this
semester: \BibTeX. You will briefly encounter it today.

\subsection*{Look at a Sample Proposal File}
The ``{\tt cs580s2015-share}'' repository contains a new
folder named ``{\tt proposal-template}.'' Review the instructions from
last Friday on how to ``pull'' this onto your own copy of the repository.

There are three files: 
\begin{itemize}
    \itemsep0in
\item
{\tt flow.eps}, an image file used to show how
figures can be included in a \LaTeX\ document; 
\item
\verb$senior_thesis_proposal.tex$, the main document;
\item
\verb$senior_thesis_proposal.bib$, a sample bibliography file
\end{itemize}

Use {\tt vim} or {\tt gvim} to edit the file
\verb$senior_thesis_proposal.tex$.  Use the ``\verb$pdflatex$'' command
to run this through the \LaTeX\ processor and then use Evince to view
the PDF file. If you see a warning message, you can ignore it and
checking to ensure that it is not serious.

Add the following to the ``Introduction'' section of the proposal template:
\begin{itemize}
\item
a display equation that uses a number of specialized
mathematical characters---Greek letters, arrows, subscript, superscripts,
etc. For example, can you recreate the following?
\[\sum_{i=1}^{\infty}\frac{1}{i^2} = \frac{\pi^2}{6}\]
\item
a numbered list of at least four items; include a nested
itemized list (``bullet list'') list as one of the items  and
a nested numbered list as one of the others (we will discuss 
the {\tt enumerate} and {\tt itemize} environments in class)
\item
a bullet list of at least four items; include a nested numbered
list as one of the items and a nested bullet list as one of the others
\item
a ``verbatim'' paragraph (we will discuss the {\tt verbatim}
environment in class)
\item
at least two paragraphs of plain text that include at least one 
non-trivial inline
mathematical formula, one or more words in fixed-width font (e.g., names 
of Java classes are often typed in fixed-width font),
one or more words in italic font (e.g., a newly-defined term 
might be italicized the first
time it appears in a paper), one or more words in bold-face font, and
two or more foreign accents (circumflex, tilde, acute accent, cedilla, etc.).


\end{itemize}


Compile it to a PDF and place it in
a new subdirectory of the repository you created last week. (For example, if you
created a repository named ``{\tt cs580S2015-<yourname>},'' you might create a
subdirectory named ``{\tt assignment6}.'') ``Add,'' ``commit,'' and ``push''
your file. If you haven't already done so, be sure to share your repository
with all of the instructors for the course.

\newpage

\subsection*{Introduction to Beamer}
\textbf{Beamer} is a \LaTeX\ document class that is used to create presentation slides.
The beamer class uses a special syntax for defining slides (known as 'frames'). 
A beamer presentation is created like any other \LaTeX\ document. 
Beamer also allows you to make 'handouts', that is the output suitable for printing, without the overlays.

\subsection*{Beamer Sample Slides}
The ``{\tt cs580s2015-share}'' repository contains a new
folder named ``{\tt slides-template}.'' It contains sample Beamer slide templates that you will
be using to create a new set of slides. There are two subdirectories in this directory:
\begin{enumerate}
\item simple: contains a basic Beamer source for creating slides
\item allegheny: contains source for creating slides that are color-customized to Allegheny College. You will notice this subdirectory contains a few .sty files that customize the layout of the slides and colors.tex file that customizes the colors. At some point you may want to play around with these files to create your own customized style for the slides, but it does not have to be today!
\end{enumerate}

\textbf{Beamer Assignment:}
Select a theme for your slides, you may not use 'default' or 'Berkeley' themes that we have used in class. Create a presentation with at least 5 slides. You slides should have the following:
\begin{itemize} 
	\item A box (example block, theorem, etc.)
	\item A table with at least two columns and two rows
	\item At least one button to jump to another slide
	\item At least one slides with creative use of overlays
	\item At least one link (to a website)
\end{itemize}

Compile your Beamer source to a PDF and place the ``.pdf'' along with
``.tex'' file in the same subdirectory of the repository in which you
placed your \LaTeX\ ``.pdf'' for the first part of this assignment.
Please see the course instructor if you have any questions.

\end{document}
