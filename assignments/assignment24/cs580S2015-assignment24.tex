%!TEX root=cs580S2015-assignment24.tex
% mainfile: cs580S2015-assignment24.tex

\input{580pre.tex}
\usepackage{ulem}
\usepackage[compact]{titlesec}

\begin{document}

\MYTITLE{Assignment 24\\Module 4 Proposal\\
Due Tuesday, 28 April}
\MYHEADERS{Assignment 24}{Due 28 April}

Write a senior thesis proposal that is related to the topic of ``computer graphics and information visualization''.
Your proposal should be a {\em minimum} of five pages long, using the \url{senior_thesis_proposal.tex} file in the
\url{cs580s2015-share/proposal-template} repository.  You will also need to create a {\tt .bib} file for your
references, downloading your sources from the ACM Digital Library.

Choose a descriptive and accurate title for your proposal. Provide a one-paragraph abstract that summarizes the
essential points of your proposal, connecting directly to the proposal's title.

You may alter the section headings in the proposal template if a different organization seems more suited to your topic,
but your proposal should still contain the same information---motivation and background, review of previous work in the
field, method of approach, evaluation strategy, schedule, etc. Obviously you do not need to create a complete solution
for the proposed problem, but you need to precisely describe the approach you would take (e.g., how your technique will
automatically add overlays to a graph) and describe how you could potentially solve the proposed problem, which method
or algorithm would you use and a justification for your choices.

You have to include at least one {\em technical diagram} or a {\em flowchart} in your proposal. The flowchart can
describe the process you are using, while a diagram could describe the technical details of your proposed solution, a
model or an algorithm that you are using. You have to use TikZ, or some other similar software package, to draw your
diagram in \LaTeX. In addition to asking the course instructor for help, you may consult
\url{http://www.texample.net/tikz/}, \url{http://www.ctan.org/pkg/pgf}, or \url{http://sourceforge.net/projects/pgf/}
for more examples and explanations. Additionally, your proposal should include at least one additional element such as
a formal statement of an algorithm, an equation with appropriate mathematical notation, or a screen shot of a program.

Your proposal should cite multiple scholarly resources for your topic and related fields---books, journal and conference
articles, technical reports, etc. The full details about these resources should be formatted using \BibTeX\ and should
appear in the ``References'' section of your paper.

\textcolor{black}{You must have your proposal read and edited by at least one person in this class, and you must read
  and edit at least one proposal other than yours.  As you are reading the module four proposal, please look out for the
  types of errors that we have examined in the {\em Bugs in Writing} and {\em Writing for Computer Science} text books.
  Submit the hard copy of the edited version (with hand-written comments by the editor) of your proposal signed by both
  the writer and the editor.  Also submit a signed hard copy of the final version of your proposal.  Your marked version
  should use a symbol to clearly indicate that you have resolved the marked concerns. Please submit both documents in
  class on Tuesday, April 28, and also commit your final {\tt .tex} and {\tt .bib} files to the your course repository
you have shared with the faculty in a subfolder with a name like ``{\tt assignment24}''.}

I'm available in person and via email if you want to run any ideas by me, or if you are having problems with your
\LaTeX. To ensure that I can offer timely assistance, please ask for help early!

\end{document}
