%!TEX root=cs580S2014-assignment7.tex
% mainfile: cs580S2014-assignment7.tex 
\input{580pre.tex}
\usepackage[compact]{titlesec}

\begin{document}

\MYTITLE{Assignment 8\\{\em Writing for Computer Science} Pr\'{e}cis\\
Due 6 February 2015}
\MYHEADERS{Assignment 8}{Due Friday, 6 Feb.}

\subsection*{What is a ``Pr\'{e}cis''?}
A pr\'{e}cis is a compact paraphrased summary of a longer work. 
For each of
modules 2, 3, and 4
you will be assigned one or more sections of Zobel's book {\em Writing 
for Computer Science} and asked to write a one-page pr\'{e}cis. 
Since the goal of the pr\'{e}cis is to capture and paraphrase
the {\em most important points} made in the original document, it forces you to think clearly about what you read.

Writing a pr\'{e}cis is similar to writing an abstract, except that an abstract usually borrows words and sentences
directly from the larger work and a pr\'{e}cis is a paraphrase. Nevertheless, writing a pr\'{e}cis is excellent practice
for writing abstracts (which you will have do do throughout this class and your entire senior year), since it forces you
to identify and focus on the essentials.

\subsection*{Write a Pr\'{e}cis}

Using \LaTeX, create a one-page pr\'{e}cis of chapters 1 and 2 in Zobel's {\em Writing for Computer Science.} If you
believe that others might have chosen differently regarding the topics to highlight, or if you are yourself in doubt
about which topics are most important, give reasons for the choices you made.

Write in a formal
scholarly style. Review what you have learned in your FS and other courses
about formal writing---write simply and objectively, paying attention to
organization, grammar, spelling, and style. You do not need to provide 
citations, footnotes, or a bibliography.

\subsection*{A Final Word}

The junior seminar is meant to simulate a real research group in a company or a university.  A member of a research
group should not feel inhibited about approaching another member of the group (or, for that matter, a faculty member)
for advice, or to discuss an interpretation of a reading, or to ask for critical feedback on a document, project, or
presentation.  Similarly, group members should always be willing to act as a sounding board or to provide feedback to
others.

\begin{quote}
\color{black}
You may (and, in fact, you are \textit{\textbf{encouraged}} to) discuss the chapters among
yourselves and try to identify the most crucial points.

You may (and, in fact, you are  \textit{\textbf{encouraged}} to) share
drafts of this and future writing assignments with one another for review
and constructive criticism. 
\end{quote}

It is possible to do this and still produce a document that is your own. The spirit of the Honor Code must still
predominate, but you should now possess the maturity to be able to distinguish when you are ``growing your ideas'' in
the fertile soil of shared discussion and criticism and when you are ``taking somebody else's work and presenting it as
your own.'' Do good work. Help one another. Please see the course instructor if you have questions about this matter!

\end{document}
