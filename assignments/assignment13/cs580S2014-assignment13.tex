%!TEX root=cs580S2014-assignment12.tex
% mainfile: cs580S2014-assignment12.tex 
%!TEX root=cs580S2015-syllabus.tex
% mainfile: cs580S2015-syllabus.tex 

% CS 580 style
% Typical usage (all UPPERCASE items are optional):
%       \input 580pre
%       \begin{document}
%       \MYTITLE{Title of document, e.g., Lab 1\\Due ...}
%       \MYHEADERS{short title}{other running head, e.g., due date}
%       \PURPOSE{Description of purpose}
%       \SUMMARY{Very short overview of assignment}
%       \DETAILS{Detailed description}
%         \SUBHEAD{if needed} ...
%         \SUBHEAD{if needed} ...
%          ...
%       \HANDIN{What to hand in and how}
%       \begin{checklist}
%       \item ...
%       \end{checklist}
% There is no need to include a "\documentstyle."
% However, there should be an "\end{document}."
%
%===========================================================

\documentclass[11pt,twoside,titlepage]{article}

\usepackage{threeparttop}
\usepackage{graphicx}
\usepackage{latexsym}
\usepackage{color}
\usepackage{listings}
\usepackage{fancyvrb}
%\usepackage{pgf,pgfarrows,pgfnodes,pgfautomata,pgfheaps,pgfshade}
\usepackage{tikz}
\usepackage[normalem]{ulem}
\tikzset{
    %Define standard arrow tip
%    >=stealth',
    %Define style for boxes
    oval/.style={
           rectangle,
           rounded corners,
           draw=black, very thick,
           text width=6.5em,
           minimum height=2em,
           text centered},
    % Define arrow style
    arr/.style={
           ->,
           thick,
           shorten <=2pt,
           shorten >=2pt,}
}
\usepackage[noend]{algorithmic}
\usepackage[noend]{algorithm}
\newcommand{\bfor}{{\bf for\ }}
\newcommand{\bthen}{{\bf then\ }}
\newcommand{\bwhile}{{\bf while\ }}
\newcommand{\btrue}{{\bf true\ }}
\newcommand{\bfalse}{{\bf false\ }}
\newcommand{\bto}{{\bf to\ }}
\newcommand{\bdo}{{\bf do\ }}
\newcommand{\bif}{{\bf if\ }}
\newcommand{\belse}{{\bf else\ }}
\newcommand{\band}{{\bf and\ }}
\newcommand{\breturn}{{\bf return\ }}
\newcommand{\mod}{{\rm mod}}
\renewcommand{\algorithmiccomment}[1]{$\rhd$ #1}
\newenvironment{checklist}{\par\noindent\hspace{-.25in}{\bf Checklist:}\renewcommand{\labelitemi}{$\Box$}%
\begin{itemize}}{\end{itemize}}
\pagestyle{threepartheadings}
\usepackage{url}
\usepackage{wrapfig}
% removing the standard hyperref to avoid the horrible boxes
%\usepackage{hyperref}
\usepackage[hidelinks]{hyperref}
% added in the dtklogos for the bibtex formatting
\usepackage{dtklogos}
%=========================
% One-inch margins everywhere
%=========================
\setlength{\topmargin}{0in}
\setlength{\textheight}{8.5in}
\setlength{\oddsidemargin}{0in}
\setlength{\evensidemargin}{0in}
\setlength{\textwidth}{6.5in}
%===============================
%===============================
% Macro for document title:
%===============================
\newcommand{\MYTITLE}[1]%
   {\begin{center}
     \begin{center}
     \bf
     CMPSC 580\\Topics and Research Methods in Computer Science\\
     Spring 2014
     \medskip
     \end{center}
     \bf
     #1
     \end{center}
}
%================================
% Macro for headings:
%================================
\newcommand{\MYHEADERS}[2]%
   {\lhead{#1}
    \rhead{#2}
    %\immediate\write16{}
    %\immediate\write16{DATE OF HANDOUT?}
    %\read16 to \dateofhandout
    \def \dateofhandout {January 14, 2014}
    \lfoot{\sc Handed out on \dateofhandout}
    %\immediate\write16{}
    %\immediate\write16{HANDOUT NUMBER?}
    %\read16 to\handoutnum
    \def \handoutnum {1}
    \rfoot{Handout \handoutnum}
   }

%================================
% Macro for bold italic:
%================================
\newcommand{\bit}[1]{{\textit{\textbf{#1}}}}

%=========================
% Non-zero paragraph skips.
%=========================
\setlength{\parskip}{1ex}

%=========================
% Create various environments:
%=========================
\newcommand{\PURPOSE}{\par\noindent\hspace{-.25in}{\bf Purpose:\ }}
\newcommand{\SUMMARY}{\par\noindent\hspace{-.25in}{\bf Summary:\ }}
\newcommand{\DETAILS}{\par\noindent\hspace{-.25in}{\bf Details:\ }}
\newcommand{\HANDIN}{\par\noindent\hspace{-.25in}{\bf Hand in:\ }}
\newcommand{\SUBHEAD}[1]{\bigskip\par\noindent\hspace{-.1in}{\sc #1}\\}
%\newenvironment{CHECKLIST}{\begin{itemize}}{\end{itemize}}

\usepackage{ulem}
\usepackage[compact]{titlesec}

\begin{document}

\MYTITLE{Assignment 12\\Module 1 Proposal\\
Due Friday, 21 February}
\MYHEADERS{Assignment 12}{Due 21 Feb.}

Write a senior thesis proposal that is related in some way to the topic of
``evolutionary computation.'' Your proposal should be a {\em minimum} of 
five pages long, using the \url{senior_thesis_proposal.tex} file in 
the \url{cs580s2014-share/proposal-template} repository on Bitbucket.
You will also need to create a {\tt .bib} file for your references (or
simply modify the file \url{senior_thesis_proposal.bib}).

Choose a descriptive and accurate title for your proposal. Provide a
one-paragraph abstract that summarizes the essential points of your proposal.

Your proposal should cite multiple scholarly resources for your thesis topic and
related fields---books, journal and conference articles, 
technical reports, etc. The full details about these resources should be 
formatted using \BibTeX\ and should appear 
in the ``References'' section of your paper.

You may alter the section headings in the proposal template if a
different organization seems more suited to your topic, but your proposal
should still contain the same information---motivation and background, 
review of previous work in the field, method of approach, evaluation strategy,
schedule, etc.

At least one portion of your proposal should include a {\em mathematical
formalization} of some  aspect of your project. This need not be as
elaborate as the formal definitions in the papers we read, but there
should be some attempt to assign variable names to the different categories
of data you will be working with, or to specify
an equation defining the fitness of a chromosome, or to make
concrete some other component of your proposal.
\begin{quote}
{\bf Example:} Let $P$ be the current population consisting of individuals
$p_0, \ldots, p_{n-1}$, where $p_i$ consists of $k$ binary digits
$b_{i,0}, \ldots, b_{i,k-1}$.
\end{quote}
You may borrow my example if it is appropriate, but you must still provide
additional mathematical formalism.

Submit a signed hard copy in class on Friday, 21 February, and
also commit your {\tt .tex} and {\tt .bib} files to the homework
repository you have shared with the faculty (preferably in a subfolder 
with a name like ``{\tt assignment12}'').
\end{document}
