%!TEX root=cs580S2014-assignment12.tex
% mainfile: cs580S2014-assignment12.tex 
\input{580pre.tex}
\usepackage{ulem}
\usepackage[compact]{titlesec}

\begin{document}

\MYTITLE{Assignment 12\\Module 1 Proposal\\
Due Friday, 21 February}
\MYHEADERS{Assignment 12}{Due 21 Feb.}

Write a senior thesis proposal that is related in some way to the topic of
``evolutionary computation.'' Your proposal should be a {\em minimum} of 
five pages long, using the \url{senior_thesis_proposal.tex} file in 
the \url{cs580s2014-share/proposal-template} repository on Bitbucket.
You will also need to create a {\tt .bib} file for your references (or
simply modify the file \url{senior_thesis_proposal.bib}).

Choose a descriptive and accurate title for your proposal. Provide a
one-paragraph abstract that summarizes the essential points of your proposal.

Your proposal should cite multiple scholarly resources for your thesis topic and
related fields---books, journal and conference articles, 
technical reports, etc. The full details about these resources should be 
formatted using \BibTeX\ and should appear 
in the ``References'' section of your paper.

You may alter the section headings in the proposal template if a
different organization seems more suited to your topic, but your proposal
should still contain the same information---motivation and background, 
review of previous work in the field, method of approach, evaluation strategy,
schedule, etc.

At least one portion of your proposal should include a {\em mathematical
formalization} of some  aspect of your project. This need not be as
elaborate as the formal definitions in the papers we read, but there
should be some attempt to assign variable names to the different categories
of data you will be working with, or to specify
an equation defining the fitness of a chromosome, or to make
concrete some other component of your proposal.
\begin{quote}
{\bf Example:} Let $P$ be the current population consisting of individuals
$p_0, \ldots, p_{n-1}$, where $p_i$ consists of $k$ binary digits
$b_{i,0}, \ldots, b_{i,k-1}$.
\end{quote}
You may borrow my example if it is appropriate, but you must still provide
additional mathematical formalism.

Submit a signed hard copy in class on Friday, 21 February, and
also commit your {\tt .tex} and {\tt .bib} files to the homework
repository you have shared with the faculty (preferably in a subfolder 
with a name like ``{\tt assignment12}'').
\end{document}
