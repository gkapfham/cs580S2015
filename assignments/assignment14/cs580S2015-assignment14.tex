%!TEX root=cs580S2014-assignment14.tex
% mainfile: cs580S2014-assignment14.tex 
\input{580pre.tex}
\usepackage{ulem}
\usepackage{tikz}
\usepackage[compact]{titlesec}

\begin{document}

\MYTITLE{Assignment 14\\Presentation Workshop, Part 2\\
Slides due Friday, 21 February\\Thesis Idea Lightning Talk, 21 February}
\MYHEADERS{Assignment 14}{Due 21 Feb.}

Using the same ``Discuss, develop, review'' process as you used on Tuesday
and all of last week, work with a partner to
complete your five-minute slide presentation to accompany your module 1 proposal.

Remember from last time:
\begin{itemize}
\item
The presentation should have the same title as your proposal and 
should contain enough slides for an intuitive and compelling introduction 
to your research idea.
\item
Use \LaTeX\ and Beamer to create your slides. Choose a pleasing Beamer
theme---don't just use the ``plain'' style.
\item
Be sure to have a title slide, a table of contents, and a
concluding slide that mentions any major references you have used.
\end{itemize}

On Friday, hand in a printed copy of your slides. If you have overlays,
be sure to change the file to read:
{\color{blue}
\begin{center}
\verb$\documentclass[handout]{beamer}$
\end{center}
}

Also upload your slides to your shared Bitbucket repository in a folder for
``assignment 14.'' We plan to post these slides on the course Web site 
for everyone to view.

%\begin{quote}
%\color{blue}NOTE: Due to the error in the syllabus, you will not be
%presenting these slides on Friday.  Instead, Friday you will present
%a ``lightning talk'' on a possible senior project idea of your own (not
%necessarily related to evolutionary computation!)
%The creation of slides to
%accompany a senior thesis proposal is an important part of your preparation
%for your ``real'' senior project.
%We will try to post all slide presentations
%online so that others may appreciate your work!
%\end{quote}

\subsection*{About Figures}
The class will begin with a short discussion of options for creating
figures. Your best long-term option is to learn how to use
TikZ/PGF, drawing commands for \LaTeX\ that produce very high-quality figures
(see \url{http://www.texample.net/tikz/} for examples and links).
Your best short-term solution is probably
 to find a drawing tool that ``works for you'' (Dia, LibreOffice Draw,
 Inkscape, xfig, \ldots) and learn to use it.

\subsection*{Adding Content: Examples and Figures}
With your editing partner, try to come up with a simple example or simple
figure to illustrate the problem you are solving. All components of the
example should be explained clearly in your text.
(This will probably help you with your proposal as well as your
presentation.)

Remember, figures don't need to be drawings---they can be tables or
algorithms. Try to avoid using images taken from papers you have read,
unless you have requested permission from the authors to use them or
unless there is a clear indication---Creative Commons language or something
similar---that the figure in question may be used. Regardless, {\em always}
cite a reference for an image taken from an external source. The attached
document on using third party material in ACM publications contains additional 
information.

Now complete your presentation slides. 

\subsection*{Try Out Your Presentation}
With your editing partner, go through your slides one at a time, asking each
other questions and making comments (``you used the term 
`fitness' on slide 3, but you never defined it'')

\subsection*{Lightning Talk}
{\color{blue}Since this module had the two lightning talks switched, with
much resulting confusion, you have two options:
\begin{itemize}
\item
You may describe an idea for a possible senior project topic that interests
you. This need not be a fully-realized proposal, but can be a description
of a problem or topic that interests you and that holds promise for a
viable senior thesis. If you select this option, you need not create a
second set of slides---a simple title slide is sufficient (but of course
you are encouraged to do more if you have time).
\item
You may present your module 1 proposal idea, together with the slides you
prepared this week in class. However, you should not select this option 
unless you are
genuinely interested in the topic. We are much more interested in learning
about topics that you are likely to pursue during your senior year, even if
the presentation is not as polished or fully fleshed out as your module
1 proposal.
\end{itemize}
}

\end{document}
