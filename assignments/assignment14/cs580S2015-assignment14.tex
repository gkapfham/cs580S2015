\input{580pre.tex}
\usepackage{ulem}
\usepackage{tikz}
\usepackage[compact]{titlesec}

\begin{document}

\MYTITLE{Assignment 14\\Presentation Workshop, Parts 1 and 2\\
Slides due Friday, 20 February\\Thesis Idea Lightning Talk (Student Idea), 20 February}
\MYHEADERS{Assignment 14}{Due 20 Feb.}

Using the same ``Discuss, develop, review'' process as you used all of last week, work with your team to complete your
five-minute slide presentation to accompany your module two proposal.

Please remember the following key points:
\begin{itemize}

  \item The presentation does not have to have the same title as your proposal; but, it should contain enough slides for
    an intuitive and compelling introduction to your research idea.

\item
Use \LaTeX\ and Beamer to create your slides. Choose a pleasing Beamer
theme or create your own by following an online tutorial---don't just use the ``plain'' style.
\item
Be sure to have a title slide, a table of contents, and a
concluding slide that mentions any major references you have used.
\end{itemize}

On Friday, hand in a printed copy of your slides. If you have overlays, be sure to change the file to read:
``\verb$\documentclass[handout]{beamer}$''.

% {\color{blue}
% \begin{center}
% \verb$\documentclass[handout]{beamer}$
% \end{center}
% }

Also upload your slides to your shared Bitbucket repository in a folder for ``{\tt assignment14}''. If possible, we plan
to post these slides on the course Web site for everyone to view.

%\begin{quote}
%\color{blue}NOTE: Due to the error in the syllabus, you will not be
%presenting these slides on Friday.  Instead, Friday you will present
%a ``lightning talk'' on a possible senior project idea of your own (not
%necessarily related to evolutionary computation!)
%The creation of slides to
%accompany a senior thesis proposal is an important part of your preparation
%for your ``real'' senior project.
%We will try to post all slide presentations
%online so that others may appreciate your work!
%\end{quote}

\subsection*{Comments about Figures}

The class will begin with a short discussion of options for creating figures. Your best long-term option is to learn how
to use TikZ/PGF, drawing commands for \LaTeX\ that produce very high-quality figures (see
  \url{http://www.texample.net/tikz/} for details and links and \url{http://www.texample.net/tikz/examples/}
  for some inspiration). Your best short-term solution is probably to find a drawing tool that ``works for you'' (e.g.,
    Dia, LibreOffice Draw, or Inkscape) and use it. 

\subsection*{Adding Content: Examples and Figures}

With your editing partners, try to come up with a simple example or simple figure to illustrate the problem you are
solving. All components of the example should be explained clearly in your text.  (This will probably help you with your
proposal as well as your presentation.)

Remember, figures don't need to be drawings---they can be tables or algorithms. Try to avoid using images taken from
papers you have read, unless you have requested permission from the authors to use them or unless there is a clear
indication---Creative Commons language or something similar---that the figure in question may be used. Regardless, {\em
always} cite a reference for an image taken from an external source. The attached document on using third party material
in ACM publications contains additional information.

Now complete your presentation slides and be ready to give your lightning talk on Friday!

% \subsection*{Try Out Your Presentation}

% With your editing partner, go through your slides one at a time, asking each other questions and making comments (``you
% used the term `test case' on slide 3, but you never defined it'')

% \subsection*{Lightning Talk}


\end{document}
