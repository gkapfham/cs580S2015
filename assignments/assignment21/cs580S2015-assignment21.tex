\input{580pre.tex}
\usepackage{ulem}
\usepackage{tikz}
\usepackage[compact]{titlesec}

\begin{document}

\MYTITLE{Assignment 21\\Presentation Workshop, Parts 1 and 2\\
Slides due Friday, 27 March\\Thesis Idea Lightning Talk (Student Idea), 24-27 March}
\MYHEADERS{Assignment 21}{Due 27 Mar.}

Using the same ``Discuss, develop, review'' process as you used all of last week, work with a new partner to complete a
five-minute slide presentation that describes a great senior thesis idea.

Please remember the following key points as you create your presentation:
\begin{itemize}

    \item The presentation does not have to be related to the theme of the module or your module proposal; but, it
    should furnish an intuitive and compelling introduction to a research idea.

    \item Use \LaTeX\ and Beamer to create your slides. Choose a pleasing Beamer theme or create your own by following
    an online tutorial---don't just use the ``plain'' style.

    \item Be sure to have a title slide, a table of contents, and a concluding slide that mentions any major references
    you have used. Also, please include a figure in your presentation slides.

\end{itemize}

On Friday, hand in a printed copy of your slides. If you have overlays, be sure to change the file to read:
``\verb$\documentclass[handout]{beamer}$'', thus ensuring that you save paper.

% {\color{blue}
% \begin{center}
% \verb$\documentclass[handout]{beamer}$
% \end{center}
% }

Also upload your slides to your shared Bitbucket repository in a folder for ``{\tt assignment21}''. If possible, we plan
to post these slides on the course Web site for everyone to view.

%\begin{quote}
%\color{blue}NOTE: Due to the error in the syllabus, you will not be
%presenting these slides on Friday.  Instead, Friday you will present
%a ``lightning talk'' on a possible senior project idea of your own (not
%necessarily related to evolutionary computation!)
%The creation of slides to
%accompany a senior thesis proposal is an important part of your preparation
%for your ``real'' senior project.
%We will try to post all slide presentations
%online so that others may appreciate your work!
%\end{quote}

\subsection*{Comments about Figures}

The class will begin with a short discussion of options for creating figures. Your best long-term option is to learn how
to use TikZ/PGF, drawing commands for \LaTeX\ that produce very high-quality figures (see
  \url{http://www.texample.net/tikz/} for details and links and \url{http://www.texample.net/tikz/examples/}
  for some inspiration). Since you have been introduced to diagram programming in \LaTeX\ in a previous module, for this
  presentation, you are required to include a diagram and you must use TikZ to program it; please see the course
  instructor if you have questions.

\subsection*{Adding Content: Examples and Figures}

With your editing partner, try to come up with a simple example or simple figure to illustrate the problem you are
solving. All components of the example should be explained clearly in your text.  (This will probably help you with your
final proposal as well as your presentation.)

Remember, figures don't need to be diagrams---they can be tables or algorithms. Try to avoid using images taken from
papers you have read, unless you have requested permission from the authors to use them or unless there is a clear
indication---Creative Commons language or something similar---that the figure in question may be used. Regardless, {\em
always} cite a reference for an image taken from an external source.  Now complete your presentation slides and be ready
to give your lightning talk on Friday! Please see the course instructor if you have questions on this assignment.

% \subsection*{Try Out Your Presentation}

% With your editing partner, go through your slides one at a time, asking each other questions and making comments (``you
% used the term `test case' on slide 3, but you never defined it'')

% \subsection*{Lightning Talk}


\end{document}
