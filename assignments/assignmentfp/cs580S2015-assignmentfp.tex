%!TEX root=cs580S2015-assignmentfp.tex
% mainfile: cs580S2015-assignmentfp.tex
%!TEX root=cs580S2015-syllabus.tex
% mainfile: cs580S2015-syllabus.tex 

% CS 580 style
% Typical usage (all UPPERCASE items are optional):
%       \input 580pre
%       \begin{document}
%       \MYTITLE{Title of document, e.g., Lab 1\\Due ...}
%       \MYHEADERS{short title}{other running head, e.g., due date}
%       \PURPOSE{Description of purpose}
%       \SUMMARY{Very short overview of assignment}
%       \DETAILS{Detailed description}
%         \SUBHEAD{if needed} ...
%         \SUBHEAD{if needed} ...
%          ...
%       \HANDIN{What to hand in and how}
%       \begin{checklist}
%       \item ...
%       \end{checklist}
% There is no need to include a "\documentstyle."
% However, there should be an "\end{document}."
%
%===========================================================

\documentclass[11pt,twoside,titlepage]{article}

\usepackage{threeparttop}
\usepackage{graphicx}
\usepackage{latexsym}
\usepackage{color}
\usepackage{listings}
\usepackage{fancyvrb}
%\usepackage{pgf,pgfarrows,pgfnodes,pgfautomata,pgfheaps,pgfshade}
\usepackage{tikz}
\usepackage[normalem]{ulem}
\tikzset{
    %Define standard arrow tip
%    >=stealth',
    %Define style for boxes
    oval/.style={
           rectangle,
           rounded corners,
           draw=black, very thick,
           text width=6.5em,
           minimum height=2em,
           text centered},
    % Define arrow style
    arr/.style={
           ->,
           thick,
           shorten <=2pt,
           shorten >=2pt,}
}
\usepackage[noend]{algorithmic}
\usepackage[noend]{algorithm}
\newcommand{\bfor}{{\bf for\ }}
\newcommand{\bthen}{{\bf then\ }}
\newcommand{\bwhile}{{\bf while\ }}
\newcommand{\btrue}{{\bf true\ }}
\newcommand{\bfalse}{{\bf false\ }}
\newcommand{\bto}{{\bf to\ }}
\newcommand{\bdo}{{\bf do\ }}
\newcommand{\bif}{{\bf if\ }}
\newcommand{\belse}{{\bf else\ }}
\newcommand{\band}{{\bf and\ }}
\newcommand{\breturn}{{\bf return\ }}
\newcommand{\mod}{{\rm mod}}
\renewcommand{\algorithmiccomment}[1]{$\rhd$ #1}
\newenvironment{checklist}{\par\noindent\hspace{-.25in}{\bf Checklist:}\renewcommand{\labelitemi}{$\Box$}%
\begin{itemize}}{\end{itemize}}
\pagestyle{threepartheadings}
\usepackage{url}
\usepackage{wrapfig}
% removing the standard hyperref to avoid the horrible boxes
%\usepackage{hyperref}
\usepackage[hidelinks]{hyperref}
% added in the dtklogos for the bibtex formatting
\usepackage{dtklogos}
%=========================
% One-inch margins everywhere
%=========================
\setlength{\topmargin}{0in}
\setlength{\textheight}{8.5in}
\setlength{\oddsidemargin}{0in}
\setlength{\evensidemargin}{0in}
\setlength{\textwidth}{6.5in}
%===============================
%===============================
% Macro for document title:
%===============================
\newcommand{\MYTITLE}[1]%
   {\begin{center}
     \begin{center}
     \bf
     CMPSC 580\\Topics and Research Methods in Computer Science\\
     Spring 2014
     \medskip
     \end{center}
     \bf
     #1
     \end{center}
}
%================================
% Macro for headings:
%================================
\newcommand{\MYHEADERS}[2]%
   {\lhead{#1}
    \rhead{#2}
    %\immediate\write16{}
    %\immediate\write16{DATE OF HANDOUT?}
    %\read16 to \dateofhandout
    \def \dateofhandout {January 14, 2014}
    \lfoot{\sc Handed out on \dateofhandout}
    %\immediate\write16{}
    %\immediate\write16{HANDOUT NUMBER?}
    %\read16 to\handoutnum
    \def \handoutnum {1}
    \rfoot{Handout \handoutnum}
   }

%================================
% Macro for bold italic:
%================================
\newcommand{\bit}[1]{{\textit{\textbf{#1}}}}

%=========================
% Non-zero paragraph skips.
%=========================
\setlength{\parskip}{1ex}

%=========================
% Create various environments:
%=========================
\newcommand{\PURPOSE}{\par\noindent\hspace{-.25in}{\bf Purpose:\ }}
\newcommand{\SUMMARY}{\par\noindent\hspace{-.25in}{\bf Summary:\ }}
\newcommand{\DETAILS}{\par\noindent\hspace{-.25in}{\bf Details:\ }}
\newcommand{\HANDIN}{\par\noindent\hspace{-.25in}{\bf Hand in:\ }}
\newcommand{\SUBHEAD}[1]{\bigskip\par\noindent\hspace{-.1in}{\sc #1}\\}
%\newenvironment{CHECKLIST}{\begin{itemize}}{\end{itemize}}

\usepackage{ulem}
\usepackage[compact]{titlesec}

\begin{document}

\MYTITLE{Assignment 25\\Final Project\\
Due Friday, 1 May by 5:00 pm}
\MYHEADERS{Assignment 25}{Due Fri., 1 May by 5:00 pm}

As the final assignment for this course, you must complete a project consisting of a poster suitable for presentation
during a department-wide poster session, a five to seven minute in-class presentation that describes your proposed
research, and a ten page final research proposal.

% introduce the final poster, perhaps the most confusing part for students

The department-wide poster session will take place on Friday, 24 April, 2015 during the normally scheduled laboratory
session.  Using technical diagrams, screen shots, equations, formal statements of algorithms, and other appropriate
components,  you are responsible for presenting a poster that clearly explains your idea for senior thesis research.
Additionally, your poster should correctly use color and layout to ensure that the idea is presented in a clear and
aesthetically pleasing manner.

While the course instructors will give you a grade for your poster and your participation in the poster session, all
students in the Department of Computer Science will be invited to provide you with written and oral feedback about both
your proposed work and the poster itself.  By 9 am on the morning of the poster session, please send a PDF of your
poster---that was programmed in \LaTeX---to the course instructor so that it can be printed at a size
suitable for display.  You should now use all feedback from the poster session to refine and extend your final project
presentation and proposal. Please see the course instructor if you have any questions about the poster session.

% talk about the final presentations

The final project presentations will take place in class on Tuesday, 28 April, 2015. Each student is required to give a
short presentation about the topic that they plan to investigate for their senior thesis.  Leveraging the content that
you developed for your poster presentation, your final project presentation should also be programmed in \LaTeX\ and use
the same aforementioned kinds of items (e.g., technical diagrams and formation statements of algorithms). However, the
presentation is different from the poster because it can use graphical overlays to progressively reveal portions of a
technical diagram or highlight certain lines of an algorithm. Overall, you should ensure that your final presentation
represents your best technical talk for the semester.  Beyond giving a presentation, each student is responsible for
answering questions posed by both the course instructor and the other students in the course.  You may use all of the
feedback from this session to improve your final project proposal. Please see the course instructor if you have
questions about the presentation.

% talk about the final proposal

You should write your final proposal using the same \LaTeX\ templates that you have used for the previous module
proposals. This proposal should describe the topic that you plan to investigate for your senior thesis. Students are
encouraged, although not required, to reuse content from their module proposals as they prepare their final proposal. If
you do incorporate material from a previous proposal into your final proposal, then please indicate what content was
reused at the end of your document. The submitted version of your final proposal should take into account the feedback
that you received during both the poster session and your final presentation. In addition to having no mistakes in
spelling or grammar, your final proposal should feature all of the key elements of good writing in computer science
(e.g., diagrams, graphs, equations, algorithms, and screen shots). Please see the course instructor if you have
any questions about \mbox{this part of the assignment}.

\end{document}
