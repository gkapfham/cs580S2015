%!TEX root=cs580S2015-formal-writing.tex
% mainfile: cs580S2015-formal-writing.tex 
\input{580pre.tex}
\usepackage[compact]{titlesec}

\begin{document}

\MYTITLE{Characteristics of Scholarly or Formal Writing}
\MYHEADERS{Scholarly Writing}{}

\subsection*{Preparing to Write Your Senior Thesis Proposal}

Here's a little experiment you might want to try.  Using a PDF viewer that permits word searching, open both of the 
research papers from our current module and search for the following words:

\begin{itemize}
\item
``you''
\item
``feel''
\item
``think''
\item
``believe''
\item
``!'' (exclamation point)
\item
``math''
\item
``terrific''
\item ``basically''
\item ``everyone knows''
  \item ``obviously''
    \item ``kinda''
\item
``great'' (used as a description of worth, e.g., ``a great experiment,'' not
as an indicator of quantity, e.g., ``a great amount of time'')
\item
any contraction (e.g., ``can't'' or ``won't'' or ``doesn't'')
\item
``lot'' (as in ``a lot of data'')


\end{itemize}

\subsection*{Formal and Informal Styles}
Authors of professional papers avoid words or constructions
that detract from the objectivity of the writing. They avoid informal
constructs such as contractions or verbal shortcuts (e.g., ``math''
for ``mathematics''), imprecise or ``casual'' usages (e.g., ``a lot
of work'' instead of the more formal sounding ``a large amount
of work'' or the more precise ``an amount of work proportional to 
the square of the input size''). They avoid addressing the reader---use
of second person pronouns---eliminating phrases such as
``You must first initialize the population''
or 
``As you can see.''

Researchers also avoid phrases such as ``we think'' or ``we feel.'' Instead
of expressing a vague opinion, they present data and draw conclusions.

In your reviews, pr\'{e}cis, and, of course, proposals, adopt the style of
the research papers you are asked to read. Present your findings as a
collection of observed facts. If you are required to critique a paper
or to express an opinion about it, write in the same objective and formal
manner---show data supporting your opinion and draw a conclusion from it.

As you continue to write paper reviews, develop your own senior thesis ideas, and give feedback to your colleagues as
they write about their own ideas, please keep in mind the following examples of bad sentences that have been rephrased.
Can you identify the ``bad'' sentences that you write and find ways to make them better? To learn more about this topic,
you are encouraged to read, on your own, additional chapters in ``Writing for Computer Science'' and ``Bugs in
Writing.'' To learn more about this topic, you are encouraged to read, on your own, additional chapters in ``Writing for
Computer Science'' and ``Bugs in Writing.''

\begin{center}
\begin{tabular}{|p{2.5in}|p{2.5in}|}
\hline
\multicolumn{1}{|c|}{\bf Bad} &
\multicolumn{1}{c|}{\bf Better}\\\hline\hline
\begin{minipage}{2.5in}
\smallskip
The authors do a great job of
explaining the genetic operators, using
lots of figures to explain things.
\smallskip
\end{minipage}
&
\begin{minipage}{2.5in}
\smallskip
The authors provide a thorough explanation of
the genetic operators, providing figures to
clarify the more difficult steps.
\smallskip
\end{minipage}
\\\hline
\begin{minipage}{2.5in}
\smallskip
I feel that the second paper didn't provide enough
intuition to explain the problem.
\smallskip
\end{minipage}
&
\begin{minipage}{2.5in}
\smallskip
The second paper describes the problem in purely
abstract terms; no examples are given.
\smallskip
\end{minipage}
\\\hline
\begin{minipage}{2.5in}
\smallskip
In the Problem Statement section they have lots of math to
explain the problem.
\smallskip
\end{minipage}
&
\begin{minipage}{2.5in}
\smallskip
The third section of the paper contains
a  precise mathematical formulation of the problem.
\smallskip
\end{minipage}
\\\hline
\begin{minipage}{2.5in}
\smallskip
I basically didn't get the section about calculating the fitness function.
\smallskip
\end{minipage}
&
\begin{minipage}{2.5in}
\smallskip
The section describing the fitness function is one of the most
difficult portions of the paper and some of the details are not clear.
\smallskip
\end{minipage}
\\\hline
\end{tabular}
\end{center}


\end{document}
