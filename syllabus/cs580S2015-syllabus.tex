%!TEX root=cs580S2015-syllabus.tex
% mainfile: cs580S2015-syllabus.tex
%!TEX root=cs580S2015-syllabus.tex
% mainfile: cs580S2015-syllabus.tex 

% CS 580 style
% Typical usage (all UPPERCASE items are optional):
%       \input 580pre
%       \begin{document}
%       \MYTITLE{Title of document, e.g., Lab 1\\Due ...}
%       \MYHEADERS{short title}{other running head, e.g., due date}
%       \PURPOSE{Description of purpose}
%       \SUMMARY{Very short overview of assignment}
%       \DETAILS{Detailed description}
%         \SUBHEAD{if needed} ...
%         \SUBHEAD{if needed} ...
%          ...
%       \HANDIN{What to hand in and how}
%       \begin{checklist}
%       \item ...
%       \end{checklist}
% There is no need to include a "\documentstyle."
% However, there should be an "\end{document}."
%
%===========================================================

\documentclass[11pt,twoside,titlepage]{article}

\usepackage{threeparttop}
\usepackage{graphicx}
\usepackage{latexsym}
\usepackage{color}
\usepackage{listings}
\usepackage{fancyvrb}
%\usepackage{pgf,pgfarrows,pgfnodes,pgfautomata,pgfheaps,pgfshade}
\usepackage{tikz}
\usepackage[normalem]{ulem}
\tikzset{
    %Define standard arrow tip
%    >=stealth',
    %Define style for boxes
    oval/.style={
           rectangle,
           rounded corners,
           draw=black, very thick,
           text width=6.5em,
           minimum height=2em,
           text centered},
    % Define arrow style
    arr/.style={
           ->,
           thick,
           shorten <=2pt,
           shorten >=2pt,}
}
\usepackage[noend]{algorithmic}
\usepackage[noend]{algorithm}
\newcommand{\bfor}{{\bf for\ }}
\newcommand{\bthen}{{\bf then\ }}
\newcommand{\bwhile}{{\bf while\ }}
\newcommand{\btrue}{{\bf true\ }}
\newcommand{\bfalse}{{\bf false\ }}
\newcommand{\bto}{{\bf to\ }}
\newcommand{\bdo}{{\bf do\ }}
\newcommand{\bif}{{\bf if\ }}
\newcommand{\belse}{{\bf else\ }}
\newcommand{\band}{{\bf and\ }}
\newcommand{\breturn}{{\bf return\ }}
\newcommand{\mod}{{\rm mod}}
\renewcommand{\algorithmiccomment}[1]{$\rhd$ #1}
\newenvironment{checklist}{\par\noindent\hspace{-.25in}{\bf Checklist:}\renewcommand{\labelitemi}{$\Box$}%
\begin{itemize}}{\end{itemize}}
\pagestyle{threepartheadings}
\usepackage{url}
\usepackage{wrapfig}
% removing the standard hyperref to avoid the horrible boxes
%\usepackage{hyperref}
\usepackage[hidelinks]{hyperref}
% added in the dtklogos for the bibtex formatting
\usepackage{dtklogos}
%=========================
% One-inch margins everywhere
%=========================
\setlength{\topmargin}{0in}
\setlength{\textheight}{8.5in}
\setlength{\oddsidemargin}{0in}
\setlength{\evensidemargin}{0in}
\setlength{\textwidth}{6.5in}
%===============================
%===============================
% Macro for document title:
%===============================
\newcommand{\MYTITLE}[1]%
   {\begin{center}
     \begin{center}
     \bf
     CMPSC 580\\Topics and Research Methods in Computer Science\\
     Spring 2014
     \medskip
     \end{center}
     \bf
     #1
     \end{center}
}
%================================
% Macro for headings:
%================================
\newcommand{\MYHEADERS}[2]%
   {\lhead{#1}
    \rhead{#2}
    %\immediate\write16{}
    %\immediate\write16{DATE OF HANDOUT?}
    %\read16 to \dateofhandout
    \def \dateofhandout {January 14, 2014}
    \lfoot{\sc Handed out on \dateofhandout}
    %\immediate\write16{}
    %\immediate\write16{HANDOUT NUMBER?}
    %\read16 to\handoutnum
    \def \handoutnum {1}
    \rfoot{Handout \handoutnum}
   }

%================================
% Macro for bold italic:
%================================
\newcommand{\bit}[1]{{\textit{\textbf{#1}}}}

%=========================
% Non-zero paragraph skips.
%=========================
\setlength{\parskip}{1ex}

%=========================
% Create various environments:
%=========================
\newcommand{\PURPOSE}{\par\noindent\hspace{-.25in}{\bf Purpose:\ }}
\newcommand{\SUMMARY}{\par\noindent\hspace{-.25in}{\bf Summary:\ }}
\newcommand{\DETAILS}{\par\noindent\hspace{-.25in}{\bf Details:\ }}
\newcommand{\HANDIN}{\par\noindent\hspace{-.25in}{\bf Hand in:\ }}
\newcommand{\SUBHEAD}[1]{\bigskip\par\noindent\hspace{-.1in}{\sc #1}\\}
%\newenvironment{CHECKLIST}{\begin{itemize}}{\end{itemize}}


\usepackage{booktabs}
\renewcommand{\arraystretch}{1.2}

\begin{document}
\MYTITLE{Syllabus}
\MYHEADERS{Syllabus}{}

% \subsection*{Instructors and Contact Information}
% \begin{description}

\vspace*{-.2in}
\subsection*{Course Instructor}
Dr.\ Gregory M.\ Kapfhammer\\
\noindent Office Location: Alden Hall 108 \\
\noindent Office Phone: +1 814-332-2880 \\
\noindent Email: \url{gkapfham@allegheny.edu} \\
\noindent Twitter: \url{@GregKapfhammer} \\
\noindent Web Site: \url{http://www.cs.allegheny.edu/sites/gkapfham/}

\vspace*{-.1in}
\subsection*{Instructor's Office Hours}

\begin{itemize}
  \itemsep -.05em
  \item Monday: 1:00 pm -- 2:00 pm (15 minute time slots)
  \item Tuesday: 2:00 pm -- 4:00 pm (15 minute time slots)
  \item Wednesday: 3:00 pm -- 5:00 pm (15 minute time slots)
  \item Thursday: 10:00 am -- 11:00 am (15 minute time slots)
  \item Friday: 1:30 -- 2:30 pm (15 minute time slots)
\end{itemize}
\vspace*{-.1in}

\noindent
To schedule a meeting with me during my office hours, please visit my Web site and click the ``Schedule'' link
in the top right-hand corner. Now, you can browse my office hours or schedule an appointment by clicking the correct
link and then reserving an open time slot.

\vspace{-.1in}
\subsection*{Course Meeting Schedule}

Lecture, Discussion, and Presentations: Tuesday and Thursday
11:00 am--12:15 pm; \\ Laboratory Session: Friday 2:30 pm --4:20 pm

\vspace{-.15in}
\subsection*{Course Catalogue Description}

\begin{quote}

  An advanced treatment of selected topics from various areas of computer science with an emphasis on appropriate
  research methods.  Practical skills are acquired in technical writing, critical reading, and presentation of technical
  literature in preparation for the senior project. One laboratory per week. Prerequisite: Completion of the computer
  science core or permission of the instructor.

\end{quote}

\vspace{-.25in}
\subsection*{Required Textbooks}

\noindent{\em On Being a Scientist: A Guide to Responsible Conduct in Research} (Third Edition).  Committee on Science,
Engineering, and Public Policy, National Academy of Sciences, National Academy of Engineering, and Institute of
Medicine. ISBN: 0309119715, 82 pages, 2009.\\ (References to the textbook are abbreviated as ``OBAS'').

\noindent{\em BUGS in Writing: A Guide to Debugging Your Prose} (Second Edition). Lyn Dupr\'e.  Addison-Wesley
Professional.  ISBN-10: 020137921X and ISBN-13: 978-0201379211, 704 pages, 1998.\\ (References to the textbook are
abbreviated as ``BIW'').

\noindent{\em Writing for Computer Science} (Second Edition).  Justin Zobel.  Springer ISBN-10: 1852338024 and ISBN-13:
978-1852338022, 270 pages, 2004. \\ (References to the textbook are abbreviated as ``WFCS'').

Along with reading the required books, you will be asked to study many additional papers from a wide variety of both
conference proceedings and journals.

\subsection*{Class Policies}

\subsubsection*{Grading}

The grade that a student receives in this class will be based on the following categories. All percentages are
approximate and, if the need to do so presents itself, it is possible for the assigned percentages to change during the
academic semester.

%The course instructor will assign all
%grades, often in consultation with the module professors.

\begin{center}
\begin{tabular}{ll}
Class Participation&10\%\\
Research Notebook and Meeting Record&10\%\\
Writing and Practical Skill Assignments&15\%\\
Research Presentations&15\%\\
Module Proposals&30\%\\
Final Project&20\%
\end{tabular}
\end{center}

Each of the above grading categories has the following definition:

\begin{itemize}

  \itemsep 0em

  \item {\em Class Participation:} All students are required to actively participate during all of the class sessions.
    Your participation will take forms such as answering questions about the required reading assignments, asking the
    presenting student(s) a constructive question, serving as a session chair for a group of presentations, and leading
    a brainstorming session. Whenever appropriate, you will receive a class participation grade for each module of the
    course.

\item {\em Research Notebook and Meeting Records}: All students must keep a research notebook throughout the semester.
  Your notebook will contain your observations about all of the reading assignments and details about your own research
  interests. Each of the dated and signed notebook entries should include paragraphs, diagrams, lists of important
  points and relevant questions, links to Web sites, description of software installation procedures, and other
  information about research in computer science. Your research notebook will be collected and graded at the end of each
  module. For every course module, students are asked to attend a fifteen minute meeting with the instructor who
  coordinates that module. After your meeting with this individual, you must record the date, time, meeting
  subject, and receive the signature of the professor. Your meeting record will be collected at the end of the semester
  and a lack of signatures will lead to a reduction of your score for this part of your grade.

\item {\em Writing and Practical Skill Assignments:} For each of the assigned research articles, a student will be
  responsible for completing a paper review containing short paragraphs that summarize and evaluate the paper and
  then propose interesting questions, insights, and areas for future work suggested by the reading. For each of the
  reading assignments from WFCS, OBAS, and BIW, you must write a precis, or a ``concise summary of essential points,
  statements, or facts'' about the assignment (Merriam-Webster Online Dictionary). During each module of the course,
  students also must complete a wide variety of practical skill assignments (e.g., writing technical papers in \LaTeX,
  using the ACM Digital Library and \BibTeX, creating technical diagrams with Graphviz and PGF/Ti\emph{k}Z, formatting
  algorithms and equations in \LaTeX, and installing open source software). Evidence that a student has mastered the
  practical skills taught during the module must be evident in the module proposal that the student submits.

\item {\em Research Presentations:} During each module of the course, a team of students will give a fifty to sixty
  minute presentation explaining the assigned article(s) and suggesting areas for future research. On the last day of
  the second and fourth weeks of a module, every student will give a lightning talk --- a short three to five minute
  presentation, leveraging no more than two slides, that effectively describes the topic. The first lightning talk must
  suggest a new research idea connected to the theme of the current module. With a topic distinct from the first talk,
  the second one should explore another idea for your senior thesis research.

\item {\em Module Proposals and Presentations:} Using \LaTeX, \BibTeX, Vim, and other relevant technical writing tools,
  a student is responsible for creating a five page proposal and a five minute presentation that both connect to the
  theme of each module. The proposal should contain an interesting and informative title, a one paragraph abstract, and
  several sections of text that describe your proposed research. The presentation should have the same title as your
  proposal and contain enough slides for a short, yet intuitive and compelling, introduction to your idea.  Students are
  encouraged to meet with the course instructors and \mbox{the module} professor about their proposed research.  The
  proposal and the presentation slides must contain evidence that the student can use all of the practical skills that
  were taught during the module.  The proposal and the presentation slides are due on the fourth Friday of \mbox{the
    module}.

  %The module professor and the course instructor will collaboratively assign a grade to each of your module proposals.

  %All professors in the Department of Computer Science will be invited to attend your research presentations and assist
  %in the assignment of the grade for this part of your final project.  The course instructor and the professor whose
  %areas of expertise are most related to your proposed research will evaluate your proposal and assign a grade.

\item {\em Final Project:} Each student must complete a final project that consists of a ten page research proposal, a
  poster suitable for presentation during a department-wide poster session, and a fifteen to twenty minute presentation
  describing your proposed research to the class. While the course instructors will give you a grade for your poster and
  your participation in the poster session, all students and faculty in the Department of Computer Science will be
  invited to provide you with written feedback about your proposed work and the poster itself.

\end{itemize}

\vspace{-.3in}
\subsubsection*{Assignment Submission}

All assignments will have a stated due date. The printed and electronic version of the assignment is to be turned in at
the beginning of the class on that due date; the printed materials must be dated and signed with the Honor Code pledge
of the student(s) completing the work.  Late assignments will be accepted for up to one week past the assigned due date
with a 15\% penalty. All late work must be submitted at the beginning of the session that is scheduled one week
after the due date. Unless special arrangements are made with the course instructors, no assignments will be accepted
after the late deadline. For any assignment completed in a group, students must also turn in a one-page document that
describes each group member's contribution to the submitted deliverables.

% All assignments will have a given due date. The printed version of the assignment is to be turned in at the beginning of
% the class on that due date. Late assignments will be accepted for up to one week past the assigned due date with a 15\%
% penalty. All late assignments must be submitted at the beginning of the class that is scheduled one week after the given
% due date. Unless special arrangements are made with the course instructors, no assignments will be accepted after the
% late deadline.

\vspace{-.20in}
\subsubsection*{Class Attendance}

% It is mandatory for all students to attend class. If you will not be able to attend a class session, then please see the
% course instructors at least one week in advance in order to describe your situation. Students who miss more than five
% unexcused classes will have their final grade in the course reduced by one letter grade. Students who miss more than ten
% unexcused classes will automatically fail the course.

It is mandatory for all students to attend all of the class and laboratory sessions. If you will not be able to attend a
session, then please see the course instructor at least one week in advance to describe your situation.  Students who
miss more than two unexcused classes or group project meetings will have their final grade in the course reduced by one
letter grade. Students who miss more than four of the aforementioned events will automatically fail the course.

\vspace{-.20in}
\subsubsection*{Laboratory Attendance}

In order to acquire the proper skills in technical writing, critical reading, and the presentation and evaluation of
technical material, it is essential for students to have hands-on experience in a laboratory. Therefore, it is mandatory
for all students to attend the laboratory sessions. If you will not be able to attend a laboratory, then please see the
one of the course instructors at least one week in advance in order to explain your situation. Students who miss more
than two unexcused laboratories will have their final grade in the course reduced by one letter grade.  Students who
miss more than four unexcused laboratories will automatically fail the course.

\vspace{-.1in}
\subsubsection*{Use of Laboratory Facilities}

Throughout the semester, we will investigate many different software tools that computer scientists use to perform,
write about, and present research in the field.  The course instructor and the department's systems administrator have
invested a considerable amount of time to ensure that our laboratories support the completion of both the laboratory
assignments and the final project.  To this end, students are required to complete all assignments and the final project
while using the department's laboratory facilities. The course instructor and the systems administrator normally do not
assist students in configuring their personal computers.

\vspace{-.10in}
\subsubsection*{Email}

Using your Allegheny College email address, the course instructors will sometimes send out announcements about matters
such as assignment clarifications or schedule changes. You must check your email at least once a day and ensure that
you can reliably send and receive emails. This class policy is based on the following statement in {\em The Compass},
the College's student handbook.

% \vspace*{-.1in}
% \newpage
\begin{quote}
``The use of email is a primary method of communication on campus. \ldots
All students are provided with a campus email account and address while
enrolled at Allegheny and are expected to check the account on a regular
basis.''
\end{quote}

\vspace*{-.30in}
\subsubsection*{Disability Services}

The Americans with Disabilities Act (ADA) is a federal anti-discrimination statute that provides comprehensive civil
rights protection for persons with disabilities.  Among other things, this legislation requires all students with
disabilities be guaranteed a learning environment that provides for reasonable accommodation of their disabilities.
Students with disabilities who believe they may need accommodations in this class are encouraged to contact Disability
Services at 332-2898.  Disability Services is part of the Learning Commons and is located in Pelletier Library.
Please do this as soon as possible to ensure that approved accommodations are implemented in a timely fashion.

\vspace{-.20in}
\subsubsection*{Honor Code}

The Academic Honor Program that governs the entire academic program at Allegheny College is described in the Allegheny
Course Catalogue.  The Honor Program applies to all work that is submitted for academic credit or to meet non-credit
requirements for graduation at Allegheny College.  This includes all work assigned for this class (e.g., examinations,
laboratory assignments, and the final project).  All students who have enrolled in the College will work under the Honor
Program.  Each student who has matriculated at the College has acknowledged the following pledge:

\vspace*{-.1in}
\begin{quote}
I hereby recognize and pledge to fulfill my responsibilities, as defined in the Honor Code, and to maintain the
integrity of both myself and the College community as a whole.
\end{quote}
\vspace*{-.15in}

\noindent It is recognized that an important part of the learning process in any course, and particularly one in
computer science, derives from thoughtful discussions with teachers and fellow students.  Such dialogue is encouraged.
However, it is necessary to distinguish carefully between the student who discusses the principles underlying a problem
with others and the student who produces assignments that are identical to, or merely variations on, someone else's
work.  While it is acceptable for students in this class to discuss their programs, technical diagrams, proposals, paper
reviews, presentations, and other items with their classmates or other individuals, deliverables that are nearly
identical to the work of others will be taken as evidence of violating the \mbox{Honor Code}.

\vspace{-.20in}
\subsection*{Module Schedule}

\subsubsection*{Overview}

This class is divided into four modules: a two-week introduction to research in \mbox{computer science and} three
four-week modules focusing on the introduction of both distinct areas of \mbox{computer science and} the practical and
conceptual skills needed to conduct research in the field.  This schedule is preliminary and, if the need to do so
presents itself, it is possible for it to change during the semester.

\noindent
In summary, this course will adhere to the following schedule during the Spring 2015 semester.

\begin{itemize}

  \item {\bf Module One}: Introduction to Research in Computer Science

    \begin{itemize}
        \item Week One: January 12 -- January 16, 2015
        \item Week Two: January 19 -- January 23, 2015
    \end{itemize}

  \item {\bf Module Two}: Software Testing and Debugging 

    \begin{itemize}
        \item Week One: January 26 -- January 30, 2015
        \item Week Two: February 2 -- February 6, 2015
        \item Week Three: February 9 -- February 13, 2015
        \item Week Four: February 16 -- February 20, 2015
    \end{itemize}

  \item {\bf Module Three}: Cooperation in Multi-Robot Systems

    \begin{itemize}
        \item Week One: February 23 -- February 27, 2015
        \item Week Two: March 2 -- March 6, 2015
        \item Week Three: March 9 -- March 13, 2015
        \item Week Four: March 23 -- March 27, 2015
    \end{itemize}

  \item {\bf Module Four}: Graphics and Information Visualization

    \begin{itemize}
        \item Week One: March 30 -- April 3, 2015
        \item Week Two: April 6 -- April 10, 2015
        \item Week Three: April 13 -- April 17, 2015
        \item Week Four: April 20 -- April 24, 2015
    \end{itemize}

    \item {\bf Last Day of Classes}: April 28, 2015
    \item {\bf No Classes}: March 16 -- March 20 and March 31, 2015

\end{itemize}

\subsubsection*{Details}

Each of the four-week modules will adhere to a schedule in which students meet on Tuesday, Thursday, and Friday to
participate in the following types of activities: group discussions, individual and team-based presentations,
brainstorming sessions, practical and conceptual skills exercises, and the creation and editing of both a module
proposal and presentations.  The following module schedule is preliminary and, if the need to do so presents itself, it
is possible for it to change.


\vspace*{.2in}
\hspace*{-.75in}
\begin{tabular}{@{} l l l l @{}}
  \toprule

\centering

  {\bf Week} & {\bf Tuesday} & {\bf Thursday} & {\bf Friday} \\

  \midrule

  {\bf W1} &

  \begin{minipage}{2in}
    {\bf Topic}: Professor presentation \\
    {\bf Due}: None
  \end{minipage} &

  \begin{minipage}{2.2in}
    {\bf Topic}: Paper review discussion \\
    {\bf Due}: Paper review
  \end{minipage} &

  \begin{minipage}{2.3in}
    {\bf Topic}: Conceptual skills (WFCS) \\
    {\bf Due}: WFCS precis
  \end{minipage} \\

  \midrule

  {\bf W2} &

  \begin{minipage}{2in}
    {\bf Topic}: Team presentation \\
    {\bf Due}: Presentation slides
  \end{minipage} &

  \begin{minipage}{2.2in}
    {\bf Topic}: Brainstorming session \\
    {\bf Due}: Five new ideas
  \end{minipage} &

  \begin{minipage}{2.3in}
    {\bf Topic}: Lighting talks (student) \\
    {\bf Due}: Presentation slides
  \end{minipage} \\

  \midrule

  {\bf W3} &

  \begin{minipage}{2in}
    {\bf Topic}: Practical skills \\
    {\bf Due}: Proposal outline
  \end{minipage} &

  \begin{minipage}{2.2in}
    {\bf Topic}: Write \& edit proposal \\
    {\bf Due}: Proposal draft
  \end{minipage} &

  \begin{minipage}{2.3in}
    {\bf Topic}: \small{Conceptual skills ({\tiny BIW} {\tiny \&} {\tiny OBAS})} \\
    {\bf Due}: BIW and OBAS precis
  \end{minipage} \\

  \midrule

  {\bf W4} &

  \begin{minipage}{2.1in}
    {\bf Topic}: Practical skills \\
    {\bf Due}: Presentation outline
  \end{minipage} &

  \begin{minipage}{2.2in}
    {\bf Topic}: Write \& edit presentation \\
    {\bf Due}: Presentation draft
  \end{minipage} &

  \begin{minipage}{2.3in}
    {\bf Topic}: Lightning talks (module) \\
    {\bf Due}: Proposal and slides
  \end{minipage} \\

  \bottomrule

\end{tabular}

\subsection*{Welcome to an Adventure in Computer Science Research}

Computer hardware and software are everywhere! Conducting research in computer science is a challenging and rewarding
activity that leads to the production of hardware, software, and scientific insights that have the potential to
positively influence the lives of many people.  As you learn more about research methods in computer science you will
also enhance your ability to effectively write and speak about a wide range of topics. At the start of this class, the
course instructors invite you to pursue this adventure in computer science research with enthusiasm and vigor.

\end{document}
